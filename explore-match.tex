\documentclass{article}
\title{Pure Exploration of Combinatorial Bandits}
\author{Shouyuan Chen}
\date{\today}
%%%%%%%%%%%%%%%%%%%%%%%%%%%%%%%%%%%%%%%%%%%%%%%%%%%%%%%%%%%%%

% Change "article" to "report" to get rid of page number on title page
\usepackage{amsmath,amsfonts,amsthm,amssymb}
\usepackage{setspace}
\usepackage{Tabbing}
\usepackage{fancyhdr}
\usepackage{lastpage} 
\usepackage{extramarks}
\usepackage{chngpage}
\usepackage{soul,color}
\usepackage{graphicx,float,wrapfig}
\usepackage{afterpage}
\usepackage{abstract}
\usepackage{hyperref}
\usepackage{natbib}
\usepackage{algpseudocode}
\usepackage{algorithm}
\usepackage{xspace}

% In case you need to adjust margins:
\topmargin=-0.45in
\evensidemargin=0in
\oddsidemargin=0in
\textwidth=6.5in
\textheight=9.0in
\headsep=0.25in

\setlength\parindent{0pt}


% Setup the header and footer
\pagestyle{fancy}
%\lhead{\LatexerName}
%\chead{\LectureClassName: \LectureTitle}
%\rhead{\LectureDate}
%\lfoot{\lastxmark}
%\cfoot{}
\rfoot{Page\ \thepage\ of\ \pageref{LastPage}}
\renewcommand\headrulewidth{0.4pt}
\renewcommand\footrulewidth{0.4pt}

\allowdisplaybreaks

%%%%%%%%%%%%%%%%%%%%%%%%%%%%%%%%%%%%%%%%%%%%%%%%%%%%%%%%%%%%%
% Some tools
\newcommand{\junk}[1]{}

\newtheorem{define}{Definition}
\newtheorem{example}{Example}
\newtheorem{lemma}{Lemma}
\newtheorem{corollary}{Corollary}
\newtheorem{theorem}{Theorem}

\newcommand{\Algorithm}{\texttt{CGapExp}\xspace}
\newcommand{\Problem}{\texttt{ExpCMAB}\xspace}
\newcommand{\Rew}{\varphi}
\newcommand{\E}{\mathbb E}

\newcommand{\M}{\mathcal M}
\newcommand{\mmatch}{\mathcal M_{\mathsf{MATCH}}}
\newcommand{\mtop}{\mathcal M_{\mathsf{TOP}m}}
\newcommand{\mbandit}{\mathcal M_{\mathsf{BANDIT}m}}

\newcommand{\diff}{\mathsf{diff}}
\newcommand{\diffvalid}{\prec}
\newcommand{\B}{\mathcal B}
\newcommand{\C}{\mathcal C}
\newcommand{\del}{\backslash}

\newcommand{\RR}{\mathbb R}

%\newcommand{\vec}[1]{\mathbf #1}

\newcommand{\Bopt}{\mathcal B_{\mathsf{opt}}}
\newcommand{\Bmatch}{\mathcal B_{\mathsf{MATCH}}}
\newcommand{\Btop}{\mathcal B_{\mathsf{TOP}m}}
\newcommand{\Bbandit}{\mathcal B_{\mathsf{BANDIT}m}}

\DeclareMathOperator{\rank}{width}
\DeclareMathOperator{\rad}{rad}
\DeclareMathOperator{\decomp}{decomp}
\DeclareMathOperator*{\argmax}{arg\,max}
\DeclareMathOperator*{\argmin}{arg\,min}
\DeclareMathOperator{\Oracle}{Oracle}

\newcommand{\out}{\mathsf{Out}}


\let\Pr\undefined
\DeclareMathOperator{\Pr}{Pr}

\newcommand{\MultiIdent}{\textbf{Multi}\xspace}
\newcommand{\Matroid}{\textbf{Matroid}\xspace}
\newcommand{\Match}{\textbf{Match}\xspace}
\newcommand{\Path}{\textbf{Path}\xspace}

\newcommand{\inn}[1]{\left\langle #1 \right\rangle}
\newcommand{\nor}[1]{\left\|#1\right\|}
\renewcommand{\vec}[1]{\boldsymbol{#1}}

\renewcommand{\odot}{\circ}

%%%%%%%%%%%%%%%%%%%%%%%%%%%%%%%%%%%%%%%%%%%%%%%%%%%%%%%%%%%%%

\begin{document}
\begin{spacing}{1.1}
\newpage

\maketitle


\section{Pure Exploration of Combinatorial Bandits}


\textbf{\Problem: problem formulation.}
Suppose that the arms are numbered $1,2,\ldots,n$.
Each arm $e\in[n]$ is associated with a reward distribution $\Rew_e$. 
We assume that all reward distributions are $B$-sub-Gaussian, i.e. ().
Notice that all distributions that are supported on $[0,B]$ are $B$-sub-Gaussian distributions [] and therefore our model subsumes the cases of bounded rewards.
Let $w(e)$ denote the expected reward of arm $e$, i.e. $w(e)=\E_{X\sim \Rew_e}[X]$.
In addition, for any set of arms $M\subseteq [n]$, we define $w(M) = \sum_{e\in M} w(e)$ as the sum of expected rewards of arms that belong to $M$.

The learning problem of pure exploration combinatorial bandit can be formalized as a game between a learner and a stochastic environment.
At the beginning of the game, the learner is given a collection of feasible sets $\M\subseteq 2^{[n]}$ which corresponds to some combinatorial problem. 
And the reward distributions $\{\Rew_e\}_{e\in[n]}$ are unknown to the learner.
Then, the game is played for multiple rounds;
on each round $t$, the learner pulls an arm $p_t\in [n]$ and observes a reward sampled from the associated reward distribution $\Rew_{p_t}$.
The game continues until certain stopping condition is satisfied, which will be specified later.
After the game finishes, the learner is asked to output a set of arms $\out \in \M$ which maximizes the sum of expected weight, i.e. $\out=M_*$, where we denote $M_*=\argmax_{M\in \M} w(M)$ to be the optimal set of arms.
For the sake of simplicity, we shall assume that the optimal set $M_*$ is unique throughout the paper.
%Notice that, if $\epsilon = 0$, then the learner is required to identify the optimal set, i.e. $\out = M_*$.



\textbf{Fixed confidence and fixed budget.} We consider two different stopping conditions of the game, which are known as \emph{fixed confidence} setting and \emph{fixed budget} setting. 
In the fixed confidence setting, the learner can stop the game at any point and her goal is to achieve a fixed confidence about the optimality of the returned set using a small number of samples (pulls).
Specifically, given a confidence parameter $\delta$, the learner need to guarantee that $\Pr[\out = M_*] \ge 1-\delta$.
The performance is evaluated by the number of pulls used by the learner.
Notice that the learner can stop the game at any point in this setting.
In the fixed budget setting, the game stops after a fixed number rounds.
The learner tries to minimize the probability of error $\Pr[\out \not= M_*]$ within these rounds.
In this case, the learner's performance is measured by the probability of error.


\textbf{Applications.} 
Our formulation of the \Problem problem covers many online learning tasks.
We consider the following applications as running examples.

\begin{itemize}
\item \MultiIdent.
\item \Match.
\item \Path.
\end{itemize}

%\textbf{Useful notations.}
%For any vector $\vec v\in \RR^n$ and any set $M \subseteq [n]$, we define $v(M) = \sum_{e\in M} v(e)$.


\section{Algorithm and Main Results}
Our main contribution is an algorithm for solving the \Problem problem.
Our algorithm 

%In this section, we describe our algorithm for pure exploration combinatorial bandit problem.
Then, we analyze the sample complexity and the probability of error of our algorithm.


%Many common combinatorial problems admit computationally efficient oracles.

\textbf{Maximization oracle.}
For most non-trivial combinatorial problems, the size of the collection of feasible sets $\M$ is exponential in $n$.
%Hence, the definition of $\M$ 
Therefore, the learning algorithm needs a succinct representation of $\M$.
In particular, we allow the learning algorithm to use a \emph{maximization oracle} which can find the optimal set $M\in \M$ when the expected reward of each arm is known.
Specifically, we assume that there exists an oracle which takes a vector $\vec v = \big(v(1),\ldots,v(n)\big)^T$ as input and returns a set $\Oracle(\vec v) = \argmax_{M\in \M} v(M)$.
It is clear that a large class of combinatorial problems admit efficient maximization oracles.


\textbf{Algorithm.} 
Our algorithm works for both fixed confidence and fixed budget settings.
In either settings, the behaviors of our algorithm only differ in the construction of confidence radius and the stopping condition.
In the following, we describe the procedure of our algorithm.
Our algorithm maintains empirical mean $\bar w_t(e)$ and confidence radius $\rad_t(e)$ for each arm $e\in[n]$ and each round $t$.
The construction of confidence radius ensures that $|w(e)-\bar w_t(e)| \le \rad_t(e)$ holds with high probability for each arm $e \in [n]$ and each round $t>0$.
At each round $t$, our algorithm accesses the maximization oracle twice. 
The first access to the oracle computes the set $M_t=\argmax_{M\in \M} \bar w_t(M)$.
Notice that $M_t$ is the ``best'' set according to the empirical means $\bar w_t$.
% which maximizes empirical means $\bar w_t$ up to  $t$.
Then, in order to explore possible refinements of $M_t$, the algorithm uses the confidence radius to compute an adjusted expectation vector $\tilde w_t$ in the following way: for each arm $e \in M_t$, $\tilde w_t(e)$ equals to the lower confidence bound $\tilde w_t(e) = \bar w_t(e)-\rad_t(e)$; and for each arm $e\not\in M_t$, $\tilde w_t(e)$ equals to the upper confidence bound $\tilde w_t(e)=\bar w_t(e)+\rad_t(e)$.
Intuitively, the adjusted expectation vector $\tilde w_t$ penalizes arms belonging to $M_t$ and encourages exploring arms out of $M_t$.
The algorithm then calls the oracle using the adjusted expectation vector $\tilde w_t$ as input, which returns another set $\tilde M_t = \argmax_{M\in \M} \tilde w_t(M)$.
The algorithm stops if $\tilde M_t= M_t$ or the budget is exhausted, i.e. $t=T$, in the fixed budget setting.
In either cases, the algorithm outputs $\out=M_t$ as result.
Otherwise, the algorithm plays the arm belonging to the symmetric difference $(\tilde M_t \del M_t) \cup (M_t \del \tilde M_t)$ with the largest confidence radius in the end of round $t$.
The pseudo-code of the algorithm is shown in Algorithm~\ref{algo:pac}. 

\begin{algorithm}[htbp]
\begin{algorithmic}[1]
\Require Confidence parameter: $\delta \in (0,1)$; Maximization oracle: $\Oracle(\cdot): \RR^n \rightarrow \M$.
\Statex \textbf{Initialize:} Play each arm $e \in [n]$ once. Initialize empirical means $\vec {\bar w}_n$ and set $T_{n}(e) \gets 1$ for all $e$.
\For{$t=n,n+1,\ldots$}
	\State $M_t \gets \Oracle(\vec {\bar w}_t)$
	\For{$e\in [n]$}
		\If {$e\in M_t$}
			\State $\tilde w_t(e) \gets \bar w_t(e)-\rad_t(e)$
		\Else
			\State $\tilde w_t(e) \gets \bar w_t(e)+\rad_t(e)$
		\EndIf
	\EndFor
	\State $\tilde M_t \gets \Oracle(\vec{\tilde w}_t)$
	\If{$\tilde M_t = M_t$}
		\State $\out \gets M_t$
		\State \textbf{return} $\out$
	\EndIf
	\State $p_t \gets \argmax_{e\in (\tilde M_t \del M_t) \cup (M_t \del \tilde M_t)} \rad_t(e)$\label{algo:step:D}
	\State Play arm $p_t$ and observe the reward
	\State Update empirical means $\vec {\bar w}_{t+1}$ using the observed reward
	\State Update number of samples: $T_{t+1}(p_t)\gets T_{t}(p_t)+1$ and $T_{t+1}(e) \gets T_{t}(e)$ for all $e\not=p_t$
	\EndFor
\end{algorithmic}
\caption{\Algorithm: Combinatorial Gap Exploration}
\label{algo:pac}
\end{algorithm}


\subsection{Analysis}
In this part, we analyze the performance of Algorithm~\ref{algo:pac} for both fixed confidence and fixed budget settings. 


\textbf{Gap.} We begin with defining a natural complexity measure of the \Problem problem. 
For each arm $e \in [n]$, we define gap $\Delta_e$ as
\begin{equation}
\label{eq:define-delta}
\Delta_e = \begin{cases}
			   w(M_*)-\max_{M\in \M: e\in M} w(M) & \text{if } e\not \in M_*, \\
			   w(M_*)-\max_{M\in \M: e\not \in M} w(M) & \text{if } e\in M_*,
			\end{cases}
\end{equation}
where we use the convention that the maximal of an empty set is $-\infty$.
By this definition of gap $\Delta_e$, for each arm $e\not\in M_*$, $\Delta_e$ represents the gap between the optimal set $M_*$ and the best set that includes arm $e$; and, for each arm $e\in M_*$, $\Delta_e$ is the sub-optimality of the best set that does not include arm $e$.
%We notice that, for many combinatorial problems, the definition Eq.~\eqref{eq:define-delta} naturally reflects the hardness of an arm.
%().
%Figure X illustrates these interpretations.
We notice that this definition resembles the definition of gaps for \MultiIdent proposed by ().



\textbf{Exchange class.} 
The analysis of our algorithm depends on certain exchange properties of combinatorial structures.
To study these properties, we introduce notions of \emph{exchange set} and \emph{exchange class} as our tools for the analysis.
We present their definitions in the following.

%An exchange set is defined as a pair of disjoint sets.

We begin with the definition of exchange set. 
We define an exchange set $b$ as an ordered pair of disjoint sets $b=(b_+,b_-)$.
Then, we define operator $\oplus$ such that, for any set $M$ and any exchange set $b=(b_+,b_-)$, we have $M\oplus b \triangleq M\del b_- \cup b_+$.
Similarly, we also define operator $\ominus$ such that $M\ominus b \triangleq M\del b_+\cup b_-$.
We call a set of exchange sets $\B$ an \emph{exchange class for $\M$} if $\B$ satisfies the following property.
Let $M$ and $M'$ be two  elements of $\M$.
Then, for any $e \in (M\del M')$, there exists an exchange set $(b_+,b_-)\in \B$ which satisfies $e\in b_-$, $b_+ \subseteq M'\del M$, $b_- \subseteq M \del M'$, $(M\oplus b) \in \M$ and $(M'\ominus b) \in \M$.
Finally, we define the \emph{width} of exchange class $\B$ as follows
$$
\rank(\B) = \max_{(b_+,b_-) \in \B} |b_+|+|b_-|.
$$

%An exchange class can be seen as a set operations that transform one feasible set to another.

Intuitively, for any feasible sets $M$ and $M'$, there exists an exchange set $(b_+,b_-)\in \B$ which can be seen as an ``operation'' that transforms $M$ one step towards $M'$: this operation generates a new feasible set $M\oplus b$ by removing elements (including $e$) from $M$ and adding elements which belongs to $M'$.
%Therefore, for any two elements $M,M'$ of $\M$, one can sequentially apply a finite number of these operations of $\B$ to transform $M$ to $M'$.it
One can chain these operations together such that, for any $M\not= M'$, there exists a sequence of exchange sets $b_1,\ldots, b_k$ of $\B$ such that $M'=M\oplus b_1 \ldots \oplus b_k$.


\junk{
Next, we investigate the exchange classes for our running examples.
For \MultiIdent problem, we can construct the exchange class as $\B_{1}=\{(\{i\},\{j\})\;|\;\forall i\in [n], j\in [n]\}$.
It is easy to verify $\B_{1}$ is an exchange class for $\M_{\MultiIdent}$: one can transform a set of $k$ elements to another by adding and removing an element for each time.
In fact, a standard result from matroid theory, called basis exchange property (see Lemma~\ref{lemma:basis-exchange-matroid} in the appendix), shows that $\B_1$ is also an exchange class for the more general \Matroid problem.
%From Lemma~\ref{lemma:basis-exchange-matroid}, we see that $\B_1$ is an exchange class for $\M_{\Matroid}$.
Next, for \Match problem, an exchange class contains all cycles of the graph $G$.
Given two matchings $M,M'$, the union $M\cup M'$ is a union of disjoint cycles [].
These cycles are known to be augmenting cycles in combinatorial optimization literature [].
Figure~Y illustrates these exchanges classes.
}

Next, we construct the exchange classes for our running examples. 
Our constructions are summarized in Lemma~\ref{lemma:example-exchange-class}.
\begin{lemma}
There exist exchange classes $\B_{\MultiIdent}$, $\B_{\Matroid}$, $\B_{\Match}$ and $\B_{\Path}$ for $\M_{\MultiIdent}$, $\M_{\Matroid}$, $\M_{\Match}$ and $\M_{\Path}$, respectively. 
These exchangese classes can be constructed as follows
\begin{enumerate}
	\item $\B_{\MultiIdent}=\big\{(\{i\},\{j\})\;|\;\forall i\in [n], j\in [n]\big\}$.
	\item $\B_{\Matroid}=\big\{(\{i\},\{j\})\;|\;\forall i\in [n], j\in [n]\big\}$.
	\item $\B_{\Match}=\big\{(C_+,C_-)\;|\; C_+\cup C_- \text{ is a cycle of }G\big\}$.
	\item $\B_{\Path}=\big\{ (P_1, P_2) \;|\;P_1,P_2\text{ are two disjoint paths of }G\text{ with same endpoints}\big\}$.
\end{enumerate}
In addition, we have $\rank(\B_{\MultiIdent})=2$, $\rank(\B_{\Matroid})=2$, $\rank(\B_{\Match})=|V|$ and $\rank(\B_{\Path})=|V|$.
\label{lemma:example-exchange-class}
\end{lemma}
The  construction for \MultiIdent problem is straightforward. 
For \Matroid problem, we leverage the basis exchange property of matroids (see Lemma~\ref{lemma:basis-exchange-matroid} in the appendix).
And for \Match and \Path problems, we appeal to the graph-theoretical properties of matchings and paths.
We illustrate these  exchanges classes in Figure~Y.
A detailed proof of Lemma~\ref{lemma:example-exchange-class} is deferred to the supplementary material.


\textbf{Main results.} 
Our main results are problem-dependent bounds of our algorithm in both fixed confidence and fixed budget settings.
First, we show that, in the fixed confidence setting, our algorithm returns the optimal set with high probability and uses at most $\tilde O\big(\sum_e \rank(\B)^2/\Delta_e^2\big)$ samples.
\begin{theorem}
Given any $\delta \in (0,1)$, any combinatorial problem $\M \subseteq 2^{[n]}$ and any vector $\vec w \in \RR^{n}$.
Assume that the reward distribution $\Rew_e$ for each arm $e\in [n]$ is  $B$-sub-Gaussian with mean $w(e)$.
Let $\B$ be an exchange class for $\M$ and let $\{\Delta_e\}_{e\in [n]}$ be the gaps defined in Eq.~\eqref{eq:define-delta}.
Set $\rad_t(e) = B\sqrt{\frac{2\log\left(\frac{4n t^2}\delta\right)}{T_e(t)}}$ for all $t > 0$ and $e\in[n]$.

Then, with probability at least $1-\delta$, Algorithm~\ref{algo:pac} returns the optimal set $\out=M_*$ and
$$
T \le O\left(\frac{1}{B^2}\cdot \sum_{e\in [n]} \frac{\rank(\B)^2}{\Delta_e^2} \log\left(\frac{1}{B^2}\cdot \frac{n}{\delta} \sum_{e\in[n]}\frac{\rank(\B)^2}{\Delta_e^2} \right)\right),
$$
where $T$ denotes the number of samples used by Algorithm~\ref{algo:pac}.
\label{theorem:main}
\end{theorem}
Theorem~\ref{theorem:main} is a general result which provides a sample complexity for any combinatorial problem $\M$.
In addition, notice that $\rank(\B_{\MultiIdent})=O(1)$.
Therefore, the sample complexity bound of our algorithm for the \MultiIdent problem is $O\big(\sum_e \Delta_e^{-2}\log(n\delta^{-1}\sum_{a}\Delta_a^{-2})\big)$.
This matches the best known problem-dependent bounds for the \MultiIdent problem due to XXX [], within logarithmic factors.
%And Theorem~\ref{algo:pac} establishes the first sample complexity bound for the pure exploration of other combinatorial bandits.


\begin{theorem}
Given any $T>0$, any combinatorial problem $\M \subseteq 2^{[n]}$ and any vector $\vec w \in \RR^{n}$.
Assume that the reward distribution $\Rew_e$ for each arm $e\in [n]$ is  $B$-sub-Gaussian with mean $w(e)$.
Let $\B$ be an exchange class for $\M$ and let $\{\Delta_e\}_{e\in [n]}$ be the gaps defined in Eq.~\eqref{eq:define-delta}.

Fix parameter $\alpha > 0$, set the confidence radius $\rad_t(e) = B\sqrt{\frac{\alpha}{T_e(t)}}$ for all arms $e\in[n]$ and all rounds $t\in [T]$.
Run Algorithm~\ref{algo:pac} in the fixed budget mode with budget $T$.
Then, the probability of error of Algorithm~\ref{algo:pac} is bounded as follows
$$
\Pr\big[\out\not=M_*\big] \le 2Tn\exp\left(-2\alpha\right),
$$
as long as $\alpha < O\left(T\left(B^2\rank(\B)^2\left(\sum_{i\in[n]}\Delta_i^{-2}\right)\right)^{-1}\right)$.

\label{theorem:main-budget}
\end{theorem}



%Many combinatorial problem $\M$ is associated with an exchange class with small width.


%From the definition, it is easy to see that if $\B$ is an exchange class, then, for any $M\not= M'$, there exists a sequence of exchange sets $b_1,\ldots, b_k$ belonging to $\B$ such that $M'=M\oplus b_1 \ldots \oplus b_k$.
%Hence, intuitively, an exchange class for $\M$ characterizes the ``operations'' of transforming an element $M\in \M$ to another element $M'\in \M$.



\section{Lower bounds}

In this part, we establish a problem-dependent lower bound on the sample complexity of the \Problem problem. 
To state our results, we first define the notion of \emph{$\delta$-correct algorithm} as follows.
For any $\delta \in (0,1)$, we call an algorithm $\mathbb A$ a $\delta$-correct algorithm if, for any expected reward $\vec w$, the probability of error of $\mathbb A$ is at most $\delta$, i.e $\Pr[M_*=\out] \le \delta$, where $\out$ is the output of algorithm $\mathbb A$.

Our next theorem shows shat, for any combinatorial problem $\M$, any expected rewards $\vec w$ and any $\delta$-correct algorithm $\mathbb A$, algorithm $\mathbb A$ must use at least $\tilde\Omega\left(\sum_{e} \frac{1}{\Delta_e^2}\right)$ samples in expectation.

\begin{theorem}
Fix any $\M\subseteq 2^{[n]}$ and any vector $\vec w \in \RR^n$.
Suppose that, for each arm $e\in [n]$, the reward distribution $\Rew_e$ is given by $\Rew_e=\mathcal N(w(e),1)$, where $\mathcal N(\mu, \sigma^2)$ denotes a Gaussian distribution with mean $\mu$ and variance $\sigma^2$. 
Then, for any $\delta \in (0,e^{-16}/4)$ and any $\delta$-correct algorithm $\mathbb A$, we have
$$
\E[T] \ge \sum_e \frac{1}{16\Delta_e^2}\log(1/4\delta),
$$
where 
$T$ denote the number of total samples used by algorithm $\mathbb A$ and
$\Delta_e$ is defined in Eq.~\eqref{eq:define-delta}.
\label{theorem:lower-bound}
\end{theorem}

Now, we compare the sample complexity of Algorithm~\ref{algo:pac} to the lower bound provided in Theorem~\ref{theorem:lower-bound} on our running examples \MultiIdent, \Matroid, \Match and \Path.
For clarity, we consider the case that $\epsilon=0$ which corresponds to the learning problem of finding the optimal set.
We see that Algorithm~\ref{algo:pac} uses at most $\tilde O(\sum_{e} \rank(\B)^2/\Delta_e^2)$ samples.
Recall that, for \MultiIdent and \Matroid problems, Lemma~X shows that $\rank(\B)=2$.
Hence, for these two problems, Algorithm~\ref{algo:pac} achieves optimal sample complexity within logarithmic factors.

On the other hand, for $\Match(V,E)$ and $\Path(V,E)$, Lemma~X indicates that $\rank(\B)=|V| \le n$.
This means that the gap between our algorithm and this lower bound is a factor of $|V|^2$.
Notice this gap only depends on the underlying combinatorial structure of $\M$ and is independent of expected rewards $\vec w$. 
This means that  the sample complexity of Algorithm~\ref{algo:pac} has an optimal dependency on the gaps $\{\Delta_e\}_{e\in[n]}$.

However, we still remain to investigate the necessity of the dependency on $\rank(\B)$ of Algorithm~\ref{algo:pac}.
To this end, we provide evidence showing that the sample complexity of any $\delta$-correct algorithm should be related to size of exchange sets.
%Furthermore, the lower bound can be improved to be dependent on the size of exchange sets.
In fact, we show that, for any optimal exchange set $b\in\Bopt$ and any $\delta$-correct algorithm, the algorithm  must spend $\tilde \Omega\left(|b|^2/w(b)^2\right)$ samples on the arms belonging to $b$.
This result is formalized in the following theorem.
\begin{theorem}
Fix any $\M\subseteq 2^{[n]}$ and any vector $\vec w \in \RR^n$.
Suppose that, for each arm $e\in [n]$, the reward distribution $\Rew_e$ is given by $\Rew_e=\mathcal N(w(e),1)$, where $\mathcal N(\mu, \sigma^2)$ denotes a Gaussian distribution with mean $\mu$ and variance $\sigma^2$. 
Fix any $\delta \in (0,e^{-16}/4)$
and any $\delta$-correct algorithm $\mathbb A$.

Then, for any $b \in \Bopt$, we have
$$
\E[T_b] \ge \frac{|b|^2}{32w(b)^2}\log(1/4\delta),
$$
where $T_b$ denotes the number of samples of arms belonging to $b$ used by algorithm $\mathbb A$.
\end{theorem}
Notice that 


\section{Extensions}




\section{Proof of Main Results}


\textbf{Notations.} 
For any set $a\subseteq [n]$, let $\vec \chi_a \in \{0,1\}^n$ denote the incidence vector of set $a \subseteq [n]$, i.e. $\chi_a(e) = 1$ if and only if $e\in a$.
For an exchange set $b=(b_+,b_-)$, we define $\vec \chi_b \triangleq \vec \chi_{b_+}- \vec \chi_{b_-}$ as the incidence vector of $b$.
We notice that $\vec \chi_b \in \{-1,0,1\}^n$.

For each round $t$, we define vector $\vec\rad_t = \big(\rad_t(1),\ldots,\rad_t(n)\big)^T$ and recall that $\vec {\bar w}_t\in \RR^n$ is the empirical mean rewards of arms up to round $t$.

Let $\vec u\in \RR^n$ and $\vec v\in \RR^n$ be two vectors.
Let $\inn{\vec u, \vec v}$ denote the inner product of $\vec u$ and $\vec v$.
We define $\vec u \odot \vec v \triangleq \big(u(1)\cdot v(1),\ldots,u(n)\cdot v(n)\big)^T$ as the element-wise product of $\vec u$ and $\vec v$.
For any $s\in \RR$, we also define $\vec u^s \triangleq \big(u(1)^s, \ldots, u(n)^s)^T$ as the element-wise exponentiation of $\vec u$.
Let $|\vec u| = \big(|u(1)|, \ldots, |u(n)|\big)^T$ denote the element-wise absolute value of $\vec u$.



\subsection{Preparatory Lemmas}
%We define $w_t(a) = \langle \vec {\bar w}_t, \vec \chi_a \rangle$ and $\rad_t(a) = \langle \vec\rad_t, \left|\vec \chi_a \right| \rangle$, where $a$ is a set or an exchange set and $\vec \chi_a$ is the incidence vector of $a$. 


\begin{lemma}
Let $M_1 \subseteq [n]$ be a set.
Let $b=(b_+,b_-)$ be an exchange set such that 
$b_-\subseteq M_1$ and $b_+ \cap M_1 = \emptyset$.
Define $M_2 = M_1 \oplus b$.
Then, we have 
$$
\vec\chi_{M_1} +\vec\chi_{b} = \vec\chi_{M_2}.
$$
\label{lemma:exchange-char}
\end{lemma}

\begin{proof}
Recall that $M_2 = M_1 \del b_- \oplus b_+$ and $b_+\cap b_-=\emptyset$.
Therefore we see that $M_2 \del M_1 = b_+$ and $M_1 \del M_2 = b_-$.
Then, we decompose $\vec\chi_{M_1}$ as $\vec\chi_{M_1}=\vec\chi_{M_1\del M_2}+\vec\chi_{M_1\cap M_2}$.
Hence, we have
\begin{align*}
   \vec\chi_{M_1}+\vec\chi_{b} &= \vec\chi_{M_1\del M_2}+\vec\chi_{M_1\cap M_2} + \vec\chi_{b_+}-\vec\chi_{b_-}\\
   							   &= \vec\chi_{M_1\cap M_2} + \vec\chi_{M_2\del M_1}\\
   							   &= \vec\chi_{M_2}.
\end{align*}
\end{proof}


\begin{lemma}
\label{lemma:exchange}
Let $\M\subseteq 2^{[n]}$ and $\B$ be an exchange class for $\M$.
Then, for any two different elements $M,M'$ of $\M$ and any $e \in (M\del M')\cup(M'\del M)$, there exists an exchange set $b=(b_+,b_-) \in \B$ such that
$e\in (b_+\cup b_-)$, $b_-\subseteq (M\del M')$, $b_+\subseteq (M'\del M)$, 
$(M\oplus b) \in \M$ and  $(M'\ominus b) \in \M$.
%In addition, the incidence vectors are related by $\vec \chi_M +\vec \chi_b = \vec \chi_{M \oplus b}$.
Moreover, if $M' = M_*$, then we have $\inn{\vec w, \vec \chi_b} \ge \Delta_e > 0$,
where $\Delta_e$ is the gap defined in Eq.~\eqref{eq:define-delta}.
\end{lemma}

\begin{proof}
We decompose our proof into two cases.

\textbf{Case (1): $e\in M\del M'$.}

By the definition of exchange class, we know that 
there exists $b=(b_+, b_-) \in \B$ which satisfies that
$e\in b_-$, $b_- \subseteq (M\del M') $, $b_+\subseteq (M' \del M)$, $(M\oplus b) \in \M$ and $(M'\ominus b) \in \M$.

Next, if $M'=M_*$, we see that $e\not \in M_*$.
Let us consider the set $M_1 = \argmax_{M': M'\in \M \wedge e\in M'} w(M')$.
Also define $M_0 = M_*\ominus b$. 
We have already proved that $M_0\in \M$.  
%Since $e\in M_1$, we see that $M_1\not= M_*$.
Combining with the fact that $e\in M_0$, we see that $w(M_0) \le w(M_1)$. 
Therefore, we obtain that
$w(M_*)-w(M_0) \ge w(M_*)-w(M_1) = \Delta_e$.
Notice that the left-hand side of the former inequality can be rewritten using Lemma~\ref{lemma:exchange-char} as follows
$$
w(M_*)-w(M_0) = \inn{\vec w, \vec \chi_{M_*}}-\inn{\vec w, \vec \chi_{M_0}} = \inn{\vec w, \vec \chi_{M_*}-\vec\chi_{M_0}}
= \inn{\vec w,\vec \chi_b}.
$$
Therefore, we obtain $\inn{\vec w,\vec \chi_b} \ge \Delta_e$.
%We decompose $\vec \chi_M = \vec \chi_{M\del M'}+\vec\chi_{M \cap M'}$.
%Recall that $\vec \chi_b = \vec \chi_{b_+}-\vec\chi_{b_-}$

\textbf{Case (2): $e\in M'\del M$.}

Using the definition of exchange class, we see that 
there exists $c=(c_+,c_-)\in \B$ such that 
$e\in c_-$, $c_-\subseteq (M'\del M)$, $c_+\subseteq (M\del M')$, $(M'\oplus c)\in \M$
and $(M\ominus c)\in \M$.

We construct $b=(b_+,b_-)$ by setting $b_+=c_-$ and $b_-=c_+$. 
Notice that, by the construction of $b$, we have $M\oplus b = M\ominus c$ and $M'\ominus b = M'\oplus c$.
Therefore, it is clear that $b$ satisfies the requirement of the lemma.


Now, suppose that $M'=M_*$. 
In this case, we have $e\in M_*$.
Consider the set $M_3 = \argmax_{M': M'\in \M \wedge e\not \in M'} w(M')$.
We see that $w(M_*)-w(M_3)=\Delta_e$.
Define $M_2 = M_* \ominus b$ and notice that  $M_2 \in \M$.
Combining with the fact that $e\not \in M_2$, we obtain that $w(M_2) \le w(M_3)$.
Hence, we have
$w(M_*)-w(M_2) \ge w(M_*)-w(M_3)=\Delta_e$.
Similar to Case (1), applying Lemma~\ref{lemma:exchange-char} again, we have
$$
\inn{\vec w,\vec \chi_b} = w(M_*)-w(M_2) \ge \Delta_e.
$$


\end{proof}

\begin{lemma}
\label{lemma:max}
Let $M$ and $M'$ be two sets. 
Then, we have 
$$ \max_{e \in (M\del M') \cup (M'\del M)} \rad_t(e) = \big\|\vec \rad_t \odot |\vec \chi_{M'} - \vec \chi_M| \big\|_\infty.$$
\end{lemma}

\begin{proof}
Notice that $\vec\chi_{M'}-\vec\chi_{M} = \vec\chi_{M'\del M}-\vec\chi_{M\del M'}$.
In addition, since $(M'\del M) \cap (M\del M') = \emptyset$, we have
$\vec \chi_{M'\del M} \odot \vec\chi_{M\del M'} = \vec 0_n$. 
Also notice that $ \vec\chi_{M'\del M}-\vec\chi_{M\del M'} \in \{-1,0,1\}^n$.
Therefore, we have
\begin{align*}
|\vec\chi_{M'\del M}-\vec\chi_{M\del M'}| 
&= (\vec\chi_{M'\del M}-\vec\chi_{M\del M'})^2\\
&=\vec\chi_{M'\del M}^2+\vec\chi_{M\del M'}^2+2\vec \chi_{M'\del M} \odot \vec\chi_{M\del M'} \\
&=\vec\chi_{M'\del M}+\vec\chi_{M\del M'}\\
& = \vec\chi_{(M' \del M) \cup (M\del M')},
\end{align*}
where the third equation follows from the fact that $\vec\chi_{M\del M'}\in \{0,1\}^n$ and $\vec\chi_{M'\del M}\in\{0,1\}^n$.
The lemma follows immediately from the fact that $\rad_t(e) \ge 0$ and  $\vec\chi_{(M\del M')\cup(M'\del M)}\in \{0,1\}^n$.
\end{proof}

\begin{lemma}
\label{lemma:vector-technical}
Let $\vec a,\vec b, \vec c \in \RR^n$ be three vectors.
Then, we have $\inn{\vec a, \vec b\odot \vec c} = \inn{\vec a\odot \vec b,\vec c}$.
\end{lemma}

\begin{proof}
We have
\begin{align*}
	\inn{\vec a,\vec b\odot \vec c} = \sum_{i=1}^n a(i) \big(b(i) c(i)\big) = \sum_{i=1}^n \big(a(i)b(i)\big)c(i) = \inn{\vec a\odot\vec b,\vec c}.
\end{align*}
\end{proof}

\begin{lemma}
Let $M_t$ and $\vec{\tilde w_t}$ be defined in Algorithm~1. 
Let $M' \in \M$ be a feasible set.
We have
$$
\tilde w_t(M')-\tilde w_t(M_t) = 
\inn{\vec{\tilde w}_t, \vec \chi_{M'}-\vec \chi_{M_t}} = \inn{\vec {\bar w}_t, \vec \chi_{M'}-\vec\chi_{M_t}}+\inn{\vec \rad_t, |\vec \chi_{M'}-\vec\chi_{M_t}|}.
$$
\label{lemma:tilde}
\end{lemma}

\begin{proof}
We begin with proving the first part.
It is easy to verify that $\vec {\tilde w_t} = \vec {\bar w}_t+ \vec \rad_t \odot (\vec 1_n-2\vec\chi_{M_t})$.
Then, we have
\begin{align}
\inn{\vec{\tilde w}_t, \vec \chi_{M'}-\vec \chi_{M_t}}
&= \inn{\vec {\bar w}_t+ \vec \rad_t \odot (1-2\vec\chi_{M_t}), \;\vec \chi_{M'}-\vec \chi_{M_t}} \nonumber \\
&= \inn{\vec {\bar w}_t,\vec \chi_{M'}-\vec \chi_{M_t}}+\inn{\vec \rad_t, (\vec 1_n-2\vec\chi_{M_t}) \odot (\vec \chi_{M'}-\vec \chi_{M_t})}
\label{eq:l-d-1}\\
&= \inn{\vec {\bar w}_t,\vec \chi_{M'}-\vec \chi_{M_t}}+\inn{\vec \rad_t, \vec\chi_{M'}-\vec\chi_{M_t}-2\vec\chi_{M_t}\odot\vec\chi_{M'}+2\vec\chi_{M_t}^2 } \nonumber\\
&= \inn{\vec {\bar w}_t,\vec \chi_{M'}-\vec \chi_{M_t}}+\inn{\vec \rad_t, \vec\chi_{M'}^2-\vec\chi_{M_t}^2-2\vec\chi_{M_t}\odot\vec\chi_{M'}+2\vec\chi_{M_t}^2 }
\label{eq:l-d-2}\\
&= \inn{\vec {\bar w}_t,\vec \chi_{M'}-\vec \chi_{M_t}}+\inn{\vec \rad_t, (\vec\chi_{M'}-\vec\chi_{M_t})^2}
\nonumber \\ \
&= \inn{\vec {\bar w}_t,\vec \chi_{M'}-\vec \chi_{M_t}}+\inn{\vec \rad_t, \big|\vec\chi_{M'}-\vec\chi_{M_t}\big|},
\label{eq:l-d-3}
\end{align}
where
Eq.~\eqref{eq:l-d-1} follows from Lemma~\ref{lemma:vector-technical};
Eq.~\eqref{eq:l-d-2} holds since $\vec \chi_{M'}\in \{0,1\}^n$ and $\vec \chi_{M_t}\in \{0,1\}^n$
and therefore $\vec\chi_{M'}=\vec\chi_{M'}^2$ and $\vec\chi_{M_t}=\vec\chi_{M_t}^2$;
and Eq.~\eqref{eq:l-d-3} follows since $\vec\chi_{M'}-\vec\chi_{M_t}\in \{-1,0,1\}^n$.

Next, recall that $\tilde M_t = \argmax_{M\in \M} \tilde w_t(M)$.
Therefore, we have $\tilde w_t(\tilde M_t) \ge \tilde w_t(M')$.
Subtracting $\tilde w_t(M_t)$ from both sides of the former inequality, we have
\begin{equation}
\tilde w_t(\tilde M_t)-\tilde w_t(M_t) \ge \tilde w_t(M')-\tilde w_t(M_t).
\label{eq:l-d-4}
\end{equation}
The lemma follows by noticing that the left-hand side of Eq.~\eqref{eq:l-d-4} equals to 
$\inn{\vec {\bar w}_t,\vec \chi_{\tilde M_t}-\vec \chi_{M_t}}+\inn{\vec \rad_t, \big|\vec\chi_{\tilde M_t}-\vec\chi_{M_t}\big|}$
and the right-hand side equals to
$\inn{\vec {\bar w}_t,\vec \chi_{M'}-\vec \chi_{M_t}}+\inn{\vec \rad_t, \big|\vec\chi_{M'}-\vec\chi_{M_t}\big|}$.
\end{proof}


\subsection{Confidence Intervals}

For all $t>0$, we define random event $\xi_t$ as follows
\begin{equation}
\xi_t = \Big\{
\forall i\in[n],\quad 
|w(i)-\bar w_t(i)| \le \rad_t(i) 
\Big\}.
\label{eq:define-xi}
\end{equation}
We notice that random event $\xi_t$ characterizes the event that the confidence bounds of all arms are valid at round $t$.

If the confidence bounds are valid, we can generalize Eq.~\eqref{eq:define-xi} to inner products as follows.
\begin{lemma}
\label{lemma:ci-property}
Given any $t>0$, assume that event $\xi_t$ as defined in Eq.~\eqref{eq:define-xi} occurs. 
Then, for any vector $\vec a \in \RR^n$, we have
$$
\big|\inn{\vec w,\vec a} - \inn{\vec {\bar w}_t, \vec a}\big| \le \inn{\vec \rad_t, |\vec a|}.
$$
\end{lemma}

\begin{proof}
Suppose that $\xi$ occurs. Then, we have
\begin{align}
\big|\inn{\vec w,\vec a} - \inn{\vec {\bar w}_t, \vec a}\big| 
&=\big|\inn{\vec w-\vec {\bar w}_t,\vec a}\big| \nonumber \\
&=\left|\sum_{i=1}^n \big(w(i)-\bar w_t(i)\big) a(i)  \right| \nonumber \\
&\le\sum_{i=1}^n \big| w(i)-\bar w_t(i)\big| |a(i)| \nonumber \\
&\le\sum_{i=1}^{n} \rad_t(i) \cdot |a(i)| \label{eq:ci-c-1}\\
&= \inn{\vec \rad_t,  |\vec a|}, \nonumber
\end{align}
where Eq.~\eqref{eq:ci-c-1} follows the definition of event $\xi_t$ in Eq.~\eqref{eq:define-xi} and the assumption that it occurs.
\end{proof}


Next, we construct the high probability confidence intervals for the fixed confidence setting.
\begin{lemma}
\label{lemma:ci}
Suppose that the reward distribution $\Rew_e$ is a $B$-sub-Gaussian distribution for all $e\in [n]$.
And if, for all $t>0$ and all $e\in [n]$, 
the confidence radius $\rad_t(e)$ is given by
$$
\rad_t(e) = B\sqrt{\frac{2\log\left(\frac{4n t^2}\delta\right)}{T_e(t)}},
$$
where $T_e(t)$ is the number of samples of arm $e$ up to round $t$.
Then, we have
$$
\Pr\left[\bigcap_{t=1}^\infty \xi_t \right] \ge 1-\delta.
$$
\end{lemma}

\begin{proof}
%We claim that, for any $t>0$, one has $\Pr[\xi_t] \ge 1- \frac{\delta}{4t^2}$.
For any $t>0$ and $e\in [n]$, notice $\Rew_e$ is a $B$-sub-Gaussian distribution with mean $w(e)$ and $w_t(e)$ is the empirical mean of $\Rew_e$ for $T_e(t)$ samples. 
Using Hoeffding's inequality (see Lemma~\ref{lemma:hoeffeding} in Section~\ref{section:technical}), we obtain
$$
\Pr\left[ \big|\bar w_t(e)-w(e) \big| \ge B\sqrt{\frac{2\log\left(\frac{4n t^2}\delta\right)}{T_e(t)}} \right] \le \frac{\delta}{2nt^2}.
$$
By union bound over all $e\in [n]$, we see that $\Pr[\xi_t] \ge 1-\frac{\delta}{2t^2}$. 
Using a union bound again over all $t>0$, we have
\begin{align*}
\Pr\left[\bigcap_{t=1}^\infty \xi_t \right] &\ge 1-\sum_{t=1}^\infty \Pr[\neg \xi_t]\\
&\ge 1-\sum_{t=1}^\infty \frac{\delta}{2t^2}\\
&= 1-\frac{\pi^2}{12}\delta \ge 1-\delta.
\end{align*}
\end{proof}


Finally, we construct the confidence intervals for the fixed budget case.
\begin{lemma}
Fix parameter $\alpha > 0$ and the number of rounds $T > 0$.
Assume that the reward distribution $\Rew_e$ is a $B$-sub-Gaussian distribution for all $e\in [n]$.
Let the confidence radius $\rad_t(e)$ of arm $e\in[n]$ and round $t>0$ be 
$\rad_t(e)=B\sqrt{\frac{\alpha}{T_e(t)}}.$
Then, we have
$$
\Pr\left[\bigcap_{t=1}^T \xi_t \right] \ge 1-2nT\exp\left(-2\alpha\right).
$$
\label{lemma:ci-budget}
\end{lemma}

\begin{proof}
For any $t>0$ and $e\in[n]$, using Hoeffding's inequality, we have
$$
\Pr\left[ \big|\bar w_t(e)-w(e) \big| \ge \rad_t(e) \right] \le 2\exp(-2\alpha).
$$
By a union bound over all arms $e\in[n]$, we see that $\Pr[\xi_t] \ge 1-2n\exp(-2\alpha)$. 
The lemma follows immediately by using union bound again over all round $t\in [T]$.
\end{proof}

\subsection{Main Lemmas}

\begin{lemma}
\label{lemma:correct}
Given any $t > 0$, assume that event $\xi_t$ (defined in Eq.~\eqref{eq:define-xi}) occurs.
Then, if Algorithm~\ref{algo:pac} terminates at round $t$, we have $M_t=M_*$.
\end{lemma}

\begin{proof}
Suppose that $M_t \not= M_*$. 
By definition, we have $w(M_*)>w(M_t)$. 
Rewriting the former inequality, we obtain that $\inn{\vec w, \vec\chi_{M_*}} > \inn{\vec w,\vec\chi_{M_t}}$.

Applying Lemma~\ref{lemma:exchange} by setting $M=M_t$ and $M'=M_*$, we see that 
there exists $b=(b_+,b_-)\in \B$ such that $(M_t \oplus b) \in \M$.
%Then, using Lemma~\ref{lemma:exchange-char}, we see that 


Now define $M_t' = M_t \oplus b$.
Recall that $\tilde M_t =\argmax_{M\in \M} \tilde w_t(M)$ and therefore $\tilde w_t(\tilde M_t) \ge \tilde w_t(M_t')$.
Hence, we have
\begin{align}
  \tilde w_t(\tilde M_t)-\tilde w_t(M_t) 
  &\ge \tilde w_t(M_t')-\tilde w_t(M_t) \nonumber \\
  &= \inn{\vec {\bar w}_t, \vec \chi_{M_t'}-\vec\chi_{M_t}}+\inn{\vec \rad_t, |\vec \chi_{M'}-\vec\chi_{M_t}|} \label{eq:lc-1}\\
  &\ge \inn{\vec w, \vec \chi_{M_t'}-\vec\chi_{M_t}} \label{eq:lc-2}\\
  &= w(M_t')-w(M_t) > 0 \label{eq:lc-3},
\end{align}
where Eq.~\eqref{eq:lc-1} follows from Lemma~\ref{lemma:tilde};
and Eq.~\eqref{eq:lc-2} follows the assumption that event $\xi$ occurs and Lemma~X;

Therefore Eq.~\eqref{eq:lc-3} shows that $\tilde w_t(\tilde M_t) > \tilde w_t(M_t)$. 
However, this contradicts to the stopping condition of Algorithm~\ref{algo:pac}: $\tilde w_t(\tilde M_t) \le \tilde w_t(M_t)$ and the assumption that the algorithm terminates on round $t$.
\end{proof}




\begin{lemma}
\label{lemma:key-technical}
Given any $t>0$ and suppose that event $\xi_t$ (defined in Eq.~\eqref{eq:define-xi})  occurs.
For any $e\in [n]$, if $\rad_t(e) < \frac{\Delta_e}{3\rank(\B)}$, then, arm $e$ will not be pulled on round $t$, i.e. $p_t\not= e$.
\end{lemma}

\begin{proof}
Suppose, in the contrary, that $p_t = e$.
By Lemma~\ref{lemma:exchange}, there exists an exchange set $c=(c_+,c_-) \in \B$
such that $e\in (c_+\cup c_-)$, $c_- \subseteq (M_t \del \tilde M_t)$, $c_+ \subseteq (\tilde M_t \del M_t)$, $(M_t\oplus c) \in \M$ and $(\tilde M_t \ominus c) \in \M$.

%Define vector 

Now, we decompose our proof into two cases.

\textbf{Case (1): $(e \in M_* \wedge e\in c_+) \vee (e \not \in M_* \wedge e\in c_-)$.}

Define $M_t' = \tilde M_t \ominus c$ and recall that $M_t' \in \M$ due to the definition of exchange class.

First, we claim that $M_t'\not= M_*$.
Suppose that $e\in M_*$ and $e\in c_+$.
Then, we see that $e\not\in M_t'$ and hence $M_t'\not=M_*$.
On the other hand, if $e\not \in M_*$ and $e\in c_-$, then $e\in M_t'$ which also means that $M_t'\not= M_*$.
Therefore we have $M_t'\not=M_*$ in either cases.

%There exists an exchange set $b=(b_+,b_-)\in \B$ such that $e\in b_+$, $b_+\subseteq (M_* \del M_t')$, $b_- \subseteq (M_t'\del M_*)$ and $M_t'\oplus b \in \M$.

Next, we apply Lemma~\ref{lemma:exchange} by setting $M=M_t'$ and $M'=M_*$.
We see that there exists an exchange set $b\in \B$ such that, $e\in (b_+\cup b_-)$, $(M_t' \oplus b) \in \M$ and
 $\inn{\vec w, \vec \chi_b} \ge \Delta_e > 0$.
 
Now, we define vectors $\vec d = \vec \chi_{\tilde M_t} - \vec \chi_{M_t}$, $\vec d_1 = \vec\chi_{M_t'}-\vec\chi_{M_t}$ and $\vec d_2 = \vec\chi_{M_t'\oplus b}-\vec\chi_{M_t}$.
%It is clear that $d\in[0,1]^n$,$d_1\in[0,1]^n$ and $d_2\in[0,1]^n$.
By the definition of $M_t'$ and Lemma~\ref{lemma:exchange}, we see that $\vec d_1 = \vec d - \vec \chi_{c}$ and $\vec d_2 = \vec d_1+\vec \chi_b = \vec d-\vec \chi_c+\vec \chi_b$.


Then, we claim that $\nor{\vec \rad_t \odot (\vec d-\vec \chi_c)}_\infty < \frac{\Delta_e}{3\rank(\B)}$.
Since $c_-\subseteq M_t$ and $c_+\cap M_t = \emptyset$, using standard set theoretical manipulations, we can show that $M_t \del \tilde M_t= (M_t \del M_t') \cup c_-$. 
Similarly, one can show that $\tilde M_t \del M_t = (M_t' \del M_t) \cup c_+$. 
This means that $\big((M_t \del M_t') \cup (M_t'\del M_t)\big) \subseteq \big((M_t \del \tilde M_t ) \cup (\tilde M_t \del M_t)\big)$.
Then, applying Lemma~Y, we obtain
\begin{align}
   \nor{\vec \rad_t \odot (\vec d-\vec\chi_c)}_\infty 
   &= \nor{\vec \rad_t \odot (\vec \chi_{M_t'} - \vec\chi_{M_t})}_\infty \nonumber\\
   &= \max_{i\in (M_t \del M_t') \cup (M_t'\del M_t) } \rad_t(i) \nonumber \\
   &\le \max_{i\in (M_t \del \tilde M_t ) \cup (\tilde M_t \del M_t)}  \rad_t(i) \nonumber \\
   & = \rad_t(e) < \frac{\Delta_e}{3\rank(\B)} \label{eq:u-c-1-0}.
\end{align}

We claim that $\nor{\vec \rad_t \odot \vec \chi_c}_\infty < \frac{\Delta_e}{3\rank(\B)}$.
Recall that, by the definition of $c$, we have
$c_+\subseteq (\tilde M_t \del M_t)$ and $c_-\subseteq (M_t\del \tilde M_t)$. 
Hence $c_+\cup c_- \subseteq  (\tilde M_t \del M_t)\cup (M_t\del \tilde M_t)$.
Since $\vec \chi_c \in [-1,1]^n$, we see that 
\begin{align}
\nor{\vec \rad_t \odot |\vec \chi_c|}_\infty &= \max_{i\in  c_+\cup c_-} \rad_t(i) \nonumber \\
									    &\le \max_{i\in  (\tilde M_t \del M_t)\cup (M_t\del \tilde M_t)} \rad_t(i)  \nonumber \\
									    &= \rad_t(e) < \frac{\Delta_e}{3\rank(\B)} \label{eq:u-c-1-0-1}.
\end{align}


Next, we claim that $\vec d \odot \vec \chi_c = |\vec \chi_c|$.
Recall that $\vec\chi_c = \vec\chi_{c_+}-\vec\chi_{c_-}$
and $\vec d = \vec \chi_{\tilde M_t}-\vec \chi_{M_t} = \vec\chi_{\tilde M_t\del M_t} - \vec\chi_{M_t\del \tilde M_t}$.
We also notice that $c_+ \subseteq (\tilde M_t \del M_t)$ and $c_- \subseteq (M_t \del \tilde M_t)$.
This implies that $c_+ \cap (M_t \del \tilde M_t) = \emptyset$ and $c_-\cap (\tilde M_t \del M_t) = \emptyset$.
Therefore, we have
\begin{align*}
\vec d \odot \vec \chi_c &= (\vec\chi_{\tilde M_t\del M_t} - \vec\chi_{M_t\del \tilde M_t})\odot(\vec\chi_{c_+}-\vec\chi_{c_-})\\
&= \vec\chi_{\tilde M_t\del M_t}\odot \vec\chi_{c_+}+
   \vec\chi_{M_t \del \tilde M_t}\odot \vec\chi_{c_-}-
   \vec\chi_{\tilde M_t\del M_t}\odot \vec\chi_{c_-}-
   \vec\chi_{M_t\del \tilde M_t}\odot \vec\chi_{c_+}\\
&= \vec\chi_{\tilde M_t\del M_t}\odot \vec\chi_{c_+}+
   \vec\chi_{M_t \del \tilde M_t}\odot \vec\chi_{c_-} \\
&= \vec\chi_{c_+}+\vec\chi_{c_-} =|\vec\chi_c|. 
\end{align*}
where the last equality holds since $c_+\cap c_- =\emptyset$.

Now, we bound  quantity $\langle \vec \rad_t, |\vec d_2| \rangle - \langle \vec \rad_t, |\vec d| \rangle$ as follows
\begin{align}
\langle \vec \rad_t, |\vec d_2| \rangle - \langle \vec \rad_t, |\vec d| \rangle &=
\langle \vec \rad_t, |\vec d_2|-|\vec d| \rangle = \left\langle \vec \rad_t, \vec d_2^2- \vec d^2 \right\rangle
\label{eq:u-c-1-1} \\
&= \left\langle \vec \rad_t, (\vec d-\vec \chi_c+\vec \chi_b)^2- \vec d^2 \right\rangle 
\nonumber \\
&= \left\langle \vec \rad_t, \vec \chi_b^2+\vec \chi_c^2-2\vec \chi_b\odot \vec \chi_c -
						  2\vec d \odot \vec \chi_c + 2\vec d\odot\vec \chi_b \right\rangle
\nonumber \\
&= \left\langle \vec \rad_t, \vec \chi_b^2 - \vec \chi_c^2+2\vec \chi_b\odot (\vec d-\vec \chi_c)\right\rangle
\label{eq:u-c-1-2}\\
&= \left\langle \vec \rad_t, |\vec\chi_b| \right\rangle
  -\left\langle \vec \rad_t, |\vec\chi_c| \right\rangle
  -2\left\langle \vec \rad_t, \vec \chi_b\odot (\vec d-\vec \chi_c) \right\rangle
  \nonumber \\
&= \left\langle \vec \rad_t, |\vec\chi_b| \right\rangle
  -\left\langle \vec \rad_t, |\vec\chi_c| \right\rangle
  -2\left\langle \vec \rad_t \odot (\vec d-\vec \chi_c), \vec \chi_b \right\rangle
  \label{eq:u-c-1-3}\\  
& \ge \left\langle \vec \rad_t, |\vec\chi_b| \right\rangle
  -\left\langle \vec \rad_t, |\vec\chi_c| \right\rangle
  -2\left\|\vec \rad_t \odot (\vec d-\vec \chi_c)\right\|_\infty \left\| \vec \chi_b\right\|_1
  \label{eq:u-c-1-4} \\
& > \left\langle \vec \rad_t, |\vec\chi_b| \right\rangle
  -\left\langle \vec \rad_t, |\vec\chi_c| \right\rangle
  -\frac{2\Delta_e}{3\rank(\B)} \|\vec\chi_b\|_1 
  \label{eq:u-c-1-5}\\
& \ge \left\langle \vec \rad_t, |\vec\chi_b| \right\rangle
  -\left\langle \vec \rad_t, |\vec\chi_c| \right\rangle
  -\frac{2\Delta_e}{3}
  \label{eq:u-c-1-6},
\end{align}
where Eq.~\eqref{eq:u-c-1-1} holds since $\vec d\in \{-1,0,1\}^n$ and $\vec d_2\in \{-1,0,1\}^n$;
Eq.~\eqref{eq:u-c-1-2} follows from the claim that $\vec d \odot \vec \chi_c = |\vec \chi_c|=\vec\chi_c^2$;
Eq.~\eqref{eq:u-c-1-3} and Eq.~\eqref{eq:u-c-1-4} follow from Lemma~\ref{lemma:vector-technical} and H\"older's inequality;
Eq.~\eqref{eq:u-c-1-5} follows from Eq.~\eqref{eq:u-c-1-0};
and Eq.~\eqref{eq:u-c-1-6} holds since $b\in\B$ and $\nor{\vec\chi_b}_1 = |b_+|+|b_-| \le \rank(\B)$.

Applying Lemma~\ref{lemma:tilde} by setting $M' = M_t' \oplus b$ and using the fact that $\tilde w_t(\tilde M_t) \ge \tilde w_t(M_t')$, we have 
\begin{align*}
\inn{\vec {\bar w}_t, \vec d}+\inn{\vec \rad_t, |\vec d|}
& = \inn{\vec {\bar w}_t, \vec \chi_{\tilde M_t}-\vec\chi_{M_t}}+\inn{\vec \rad_t, |\vec \chi_{\tilde M_t}-\vec\chi_{M_t}|}\\
& = \tilde w_t(\tilde M_t)- \tilde w_t(M_t)\\
& \ge \tilde w_t(M_t') - \tilde w_t(M_t)\\
&= \inn{\vec {\bar w}_t, \vec \chi_{M_t'}-\vec\chi_{M_t}}+\inn{\vec \rad_t, |\vec \chi_{M_t'}-\vec\chi_{M_t}|}\\
&= \inn{\vec {\bar w}_t, \vec d_2}+\inn{\vec \rad_t, |\vec d_2|} \\
									  &= \inn{\vec {\bar w}_t, \vec d}-\inn{\vec {\bar w}_t, \vec \chi_c}+\inn{\vec {\bar w}_t,\vec\chi_b}+\inn{\vec \rad_t, |\vec d_2|},
\end{align*}
where the last equality follows from the fact that $\vec d_2 = \vec d-\vec \chi_{c}+\vec \chi_{b}$.
Rearranging the above inequality, we obtain
\begin{align}
\inn{\vec {\bar w}_t, \vec \chi_c} &\ge \inn{\vec {\bar w}_t, \vec \chi_b}+\inn{\vec \rad_t, |\vec d_2|}-\inn{\vec \rad_t, |\vec d|}\nonumber \\
&\ge  \inn{\vec {\bar w}_t, \vec \chi_b}+
\left\langle \vec \rad_t, |\vec\chi_b| \right\rangle
  -\left\langle \vec \rad_t, |\vec\chi_c| \right\rangle
  -\frac{2\Delta_e}{3} \label{eq:u-c-1-7}\\
&> \inn{\vec w, \vec \chi_b}-\inn{\vec \rad_t, \vec \chi_c}-\frac{2\Delta_e}{3} \label{eq:u-c-1-8}\\
&> \inn{\vec w, \vec \chi_b}-\frac{\Delta_e}{3}-\frac{2\Delta}{3} \label{eq:u-c-1-9}\\
&= \inn{\vec w, \vec \chi_b}-\Delta_e \ge 0,
\end{align}
where Eq.~\eqref{eq:u-c-1-7} uses Eq.~\eqref{eq:u-c-1-6}; 
Eq.~\eqref{eq:u-c-1-8} follows from the assumption that event $\xi$ occurs and Lemma~X;
and Eq.~\eqref{eq:u-c-1-8} holds since Eq.~\eqref{eq:u-c-1-0-1}.

We have shown that $\inn{\vec {\bar w}_t,\vec \chi_c}>0$. Now we can bound $\bar w_t(M_t')$ as follows
\begin{align*}
 \bar w_t(M_t') &= \inn{\vec {\bar w}_t, \vec \chi_{M_t'}} = \inn{\vec {\bar w}_t, \vec \chi_{M_t}+\vec \chi_c} =
 \inn{\vec {\bar w}_t, \vec \chi_{M_t}}+\inn{\vec {\bar w}_t, \vec \chi_c} > \inn{\vec {\bar w}_t, \vec \chi_{M_t}} = w_t(M_t).
\end{align*}
However, the definition of $M_t$ ensures that $M_t = \argmax_{M\in\M} \bar w_t(M)$, i.e. $\bar w_t(M_t) \ge \bar w_t(M_t')$. Contradiction.

\textbf{Case (2): $(e \in M_* \wedge e\in c_-) \vee (e \not \in M_* \wedge e\in c_+)$.}

First, we claim that $\tilde M_t \not= M_*$.
Suppose that $e\in M_*$ and $e\in c_-$.
Then, we see that $e\not\in \tilde M_t$, which implies that $\tilde M_t\not=M_*$.
If $e\not \in M_*$ and $e\in c_+$, then $e\in \tilde M_t$, which also implies that $\tilde M_t\not= M_*$.
Therefore we have $\tilde M_t\not=M_*$ in either cases.


Hence, by Lemma~\ref{lemma:exchange}, there exists an exchange set $b=(b_+,b_-)\in \B$ such that 
$e \in (b_+ \cup b_-)$, $b_-\subseteq  (\tilde M_t \del M_*)$, $b_+ \subseteq (M_* \del \tilde M_t)$ and
$(\tilde M_t \oplus b) \in \M$.
Lemma~\ref{lemma:exchange} also indicates that $\inn{\vec w, \vec \chi_b} \ge \Delta_e > 0$.

Next, we define vectors $\vec d = \vec \chi_{\tilde M_t} - \vec \chi_{M_t}$ and $\vec d_1 = \vec\chi_{\tilde M_t\oplus b}-\vec\chi_{M_t}$.
Notice that Lemma~\ref{lemma:exchange} gives that $\vec d_1= \vec d+\vec b$.

Then, we apply Lemma~\ref{lemma:max} by setting $M = M_t$ and $M' = \tilde M_t$. 
This shows that 
\begin{equation}
\nor{\vec \rad_t\odot \vec d}_\infty \le \max_{i: (\tilde M_t \del M_t)\cup (M_t\del \tilde M_t)} \rad_t(i) = \rad_t(e) < \frac{\Delta_e}{3}.
\label{eq:u-c-2-0}
\end{equation}

Now, we bound quantity $\inn{\vec {\bar w}_t, \vec d_1}+\inn{\vec \rad_t, |\vec d_1|}
-\inn{\vec {\bar w}_t, \vec d}-\inn{\vec \rad_t,  |\vec d|}$ as follows
\begin{align}
\inn{\vec {\bar w}_t, \vec d_1}+\inn{\vec \rad_t, |\vec d_1|}
-\inn{\vec {\bar w}_t, \vec d}-\inn{\vec \rad_t,  |\vec d|}
& = \inn{\vec {\bar w}_t, \vec \chi_b} + \inn{\vec \rad_t, |\vec d_1|-|\vec d|} \nonumber\\
& =	\inn{\vec {\bar w}_t, \vec \chi_b} + \inn{\vec \rad_t, \vec d_1^2-\vec d^2} \label{eq:u-c-2-1} \\
& =	\inn{\vec {\bar w}_t, \vec \chi_b} + \inn{\vec \rad_t, 2\vec d\odot \vec \chi_b +\vec \chi_b^2} \label{eq:u-c-2-2} \\
& =	\inn{\vec {\bar w}_t, \vec \chi_b} + \inn{\vec \rad_t, \vec\chi_b^2} + 2\inn{\vec \rad_t\odot \vec d, \vec \chi_b} \nonumber \\
& \ge \inn{\vec w, \vec \chi_b}- 2\inn{\vec \rad_t\odot \vec d, \vec \chi_b}  \label{eq:u-c-2-3} \\
& \ge \inn{\vec w, \vec \chi_b}-2\nor{\vec \rad_t\odot \vec d}_\infty\nor{\vec\chi_b}_1 \label{eq:u-c-2-3.5} \\
& > \inn{\vec w, \vec \chi_b}-\frac{2\Delta_e}{3} \label{eq:u-c-2-4} \\
& \ge 0 \label{eq:u-c-2-5},
\end{align}
where Eq.~\eqref{eq:u-c-2-1} follows from the fact that $\vec d_1\in \{-1,0,1\}^n$ and $\vec d \in \{-1,0,1\}^n$;
Eq.~\eqref{eq:u-c-2-2} holds since $\vec d_1=\vec d+\vec \chi_b$;
Eq.~\eqref{eq:u-c-2-3} follows from the assumption that $\xi$ occurs and Lemma~X;
Eq.~\eqref{eq:u-c-2-3.5} follows from Lemma~\ref{lemma:vector-technical}  and H\"older's inequality;
and Eq.~\eqref{eq:u-c-2-4} is due to Eq.~\eqref{eq:u-c-2-0}.

Therefore, we have proved that $\inn{\vec {\bar w}_t, \vec d}+\inn{\vec \rad_t,  |\vec d|} < \inn{\vec {\bar w}_t, \vec d_1}+\inn{\vec \rad_t, |\vec d_1|}.
$
However, Lemma~\ref{lemma:tilde} shows that 
\begin{align*}
\inn{\vec {\bar w}_t, \vec d}+\inn{\vec \rad_t,  |\vec d|} 
&= \inn{\vec {\bar w}_t, \vec \chi_{\tilde M_t} - \vec \chi_{M_t}}+\inn{\vec \rad_t,  |\vec \chi_{\tilde M_t} - \vec \chi_{M_t}|} \\
&= \tilde w_t(\tilde M_t) - \tilde w_t(M_t)\\
& \ge \tilde w_t(\tilde M_t \oplus b)-\tilde w_t(M_t)\\
&= \inn{\vec {\bar w}_t, \vec \chi_{\tilde M_t \oplus b} - \vec \chi_{M_t}}+\inn{\vec \rad_t,  |\vec \chi_{\tilde M_t \oplus b} - \vec \chi_{M_t}|} \\
&= \inn{\vec {\bar w}_t, \vec d_1}+\inn{\vec \rad_t,  |\vec d_1|}. 
\end{align*}
This is a contradiction and therefore $p_t\not= e$.

\end{proof}


\subsection{Proof of Theorem~\ref{theorem:main}}

Theorem~\ref{theorem:main} is now a straightforward corollary of Lemma~\ref{lemma:correct} and Lemma~\ref{lemma:key-technical}.
\begin{proof}
Lemma~\ref{lemma:ci} indicates that the event $\xi \triangleq \bigcap_{t=1}^\infty \xi_t$ occurs with probability at least $1-\delta$.
In the rest of the proof, we shall assume that this event holds.

By Lemma~\ref{lemma:correct} and the assumption on $\xi$, we see that $\out=M_*$.
Next, we focus on bounding $T$.


Let $T_e$ denote the total number of pull of arm $e\in [n]$.
Define 
$$
t_e^* = \left\lceil\frac{275}{B^2}\cdot\frac{\rank(\B)^2}{\Delta_e^2}\log\left(\frac{1}{B^2}\cdot\frac{4n}{\delta}\sum_{i\in[n]}\frac{\rank(\B)^2}{\Delta_i^2}\right)\right\rceil.
$$
We claim that, for any $e \in [n]$, we have $T_e \le \lceil t_e^* \rceil$.
Suppose that $T_e \ge \lceil t_e^* \rceil+1$.
Then, consider the round $t$ that such that $p_t=e$ and $T_e(t) =  \lceil t_e^* \rceil+1$.
Notice that $t \le T_e(t)$.
Then, by straightforward computations, we obtain
\begin{align*}
\rad_t(e) &=   B\sqrt{\frac{2\log\left(\frac{4n t^2}\delta\right)}{T_e(t)}} \le
 B\sqrt{\frac{2\log\left(\frac{4n (t_e^*)^2}\delta\right)}{t_e^*}} <  \frac{\Delta_e}{3\rank(\B)}. 	
\end{align*}
However, by Lemma~\ref{lemma:key-technical}, we see that $p_t\not=e$. This is a contradiction, and therefore the claim that $T_e \le \lceil t_e^* \rceil$ is true.

Then, the total number of samples $T$ can be bounded by
\begin{align*}
T = \sum_{e\in [n]} T_e \le \sum_{e\in [n]} \lceil t_e^* \rceil \le 
O\left(\frac{1}{B^2}\sum_{e\in [n]} \frac{\rank(\B)^2}{\Delta_e^2} \log\left(\frac{1}{B^2}\cdot \frac{n}{\delta} \sum_{e\in[n]}\frac{\rank(\B)^2}{\Delta_e^2} \right)\right).
\end{align*}

\end{proof}


\subsection{Proof of Theorem~\ref{theorem:main-budget}}

\begin{proof}
Define random event $\xi=\bigcap_{t=1}^T \xi_t$. 
By Lemma~\ref{lemma:ci-budget}, we see that $\Pr[\xi] \ge 1-2nT\exp(-2\alpha)$.
In the rest of the proof, we assume that $\xi$ happens.

We claim that Algorithm~\ref{algo:pac} before round $T$.
If the claim is true, then there exists a round $t^* < T$ such that $\tilde M_{t^*} = M_{t^*}$ and $\out = M_{t^*}$.
By assumption on $\xi$ and Lemma~\ref{lemma:correct}, we know that $M_{t^*} = M_*$.
Therefore the theorem follows immediately from this claim and the bound of $\Pr[\xi]$.

Next, we show that this claim is true.
For any round $t \in [T]$ and any arm $e\in[n]$, 
by Lemma~\ref{lemma:key-technical}, 
we see that if $\rad_t(e) < \frac{\Delta_e}{3\rank(\B)}$, then arm $e$ will no longer be pulled.
Also notice that, by construction in the fixed budget setting, the radius $\rad_t(e)$ is monotone decreasing with respect to $T_e(t)$.
Therefore, plugging the definition of $\rad_t(e)$, we have
$$
T_e \le \frac{9B^2\rank(\B)^2}{\Delta_e^2}\cdot\alpha.
$$

Summing up $T_e$ for all $e\in [n]$, we have
$$
t^* = \sum_{e\in [n]} T_e \le \alpha \cdot 9B^2\rank(\B)^2 \left(\sum_{e\in[n]}\Delta_e^{-2}\right) < T,
$$
where we have used the assumption that $\alpha < 9 T \cdot \left(B^2\rank(\B)^2 \left(\sum_{e\in[n]}\Delta_e^{-2}\right)\right)^{-1}.$

\end{proof}



\section{Proof of Lower Bounds}

\begin{lemma}
$$
\Delta_e = \min_{b: e \in b, b \in \Bopt} w(b).
$$
\end{lemma}

\begin{proof}
\end{proof}


\begin{proof}
Fix $\delta >0$, $\vec w =\{w(1),\ldots,w(n)\}$ and a $\delta$-correct policy $\mathbb A$.
For each $e\in [n]$, assume that the reward distribution is given by $\Rew_e=\mathcal N(w(e),1)$.
For any $e\in [n]$, let $T_e$ denote the  number of trials of arm $e$ used by algorithm $\mathbb A$.
In the rest of the proof, we will show that for any $e\in [n]$, the number of trials of arm $e$ is lower-bounded by
\begin{equation}
\E[T_e] \ge \frac{1}{16\Delta_e^2}\log(1/4\delta).
\label{eq:lower-each}
\end{equation}
Notice that the theorem follows immediately by summing up Eq.~\eqref{eq:lower-each} for all $e\in[n]$.


Fix an arm $e\in [n]$. We now focus on proving Eq.~\eqref{eq:lower-each}.
Consider two hypothesis $H_0$ and $H_1$. 
Under hypothesis $H_0$, all reward distributions are same with our assumption before
$$
H_0: \Rew_l = \mathcal N(w(l),1) \quad \text{for all } l \in [n].
$$
Under hypothesis $H_1$, we change the means of reward distributions such that 
$$
H_1: 
	\Rew_e = \begin{cases}
	\mathcal N(w(e)-2\Delta_e,1) & \text{if } e\in M_*\\
	\mathcal N(w(e)+2\Delta_e,1) & \text{if } e\not\in M_*
\end{cases} 
\quad\text{and } \Rew_l=\mathcal N(w(l), 1) \quad\text{for all } l\not = e.
$$

Define $M_e$ be the ``next-to-optimal'' set as follows 
$$
M_e = \begin{cases}
		 \argmax_{M\in \M: e \in M} w(M) & \text{if } e\not \in M_*, \\
	     \argmax_{M\in \M: e \not\in M} w(M) & \text{if } e\in M_*.
	  \end{cases}
$$
By definition of $\Delta_e$, we know that $w(M_*)-w(M_e)=\Delta_e$.

Let $\vec w_0$ and $\vec w_1$ be expected reward vectors under $H_0$ and $H_1$ respectively.
Notice that $w_0(M_*)-w_0(M_e)=\Delta_e > 0$.
On the other hand, 
$w_1(M_*)-w_1(M_e) = -\Delta < 0$.
This means that under $H_1$, $M_*$ is not the optimal set.
For $l\in \{0,1\}$, we use $\E_l$ and $\Pr_l$ to denote the expectation and probability, respectively, under the hypothesis $H_l$.

Define $\theta=4\delta$. Define
\begin{equation}
t_e^* = \frac{1}{16\Delta^2_e}\log\left(\frac{1}{\theta}\right).
\label{eq:define-tstar}
\end{equation}

Recall that $T_e$ denotes the total number of samples of arm $e$.
Define the event
$\mathcal A = \{T_e \le 4t_e^* \}$.

First, we show that $\Pr_0[\mathcal A] \ge 3/4$. 
This can be proved by Markov inequality as follows.
\begin{align*}
\Pr_0[T_e > 4t_e^*] &\le \frac{\E_0[T_e]}{4t_e^*} \\
					  &= \frac{t_e^*}{4t_e^*} = \frac14.
\end{align*}

Let $X_1,\ldots,X_{T_e}$ denote the sequence of reward outcomes of arm $e$.
We define $K_t(e)$ as the sum of outcomes of arm $e$ up to round $t$, i.e. $K_t(e) = \sum_{i\in [t]} X_i. $
Next, we define the event 
$$
\mathcal C=\left\{\max_{1\le t \le 4t_e^*} \left|K_t(e)-t\cdot w(e)\right|  < \sqrt{t_e^*\log(1/\theta)} \right\}.
$$
We now show that $\Pr_0[\mathcal C] \ge 3/4$.
First, notice that $K_t(e)-p_e t$ is a martingale under $H_0$.
Then, by Kolmogorov's inequality, we have
\begin{align*}
\Pr_0\left[\max_{1\le t \le 4t_e^*} \left|K_t(e)-t\cdot w(e)\right| \ge \sqrt{t_e^*\log(1/\theta)} \right]
&\le \frac{\E_0[ (K_{4t_e^*}(e)-4w(e)t_e^*)^2]}{t_e^*\log(1/\theta)}\\
&= \frac{4t_e^*}{t_e^*\log(1/\theta)}\\
&< \frac14,
\end{align*}
where the second inequality follows from the fact that $\E_0[(K_{4t_e^*}(e)-4w(e)t_e^*)^2] = 4t_e^*$; the last inequality follows 
since $\theta < e^{-16}$.

Then, we define the event $\mathcal B$ as the event that the algorithm eventually returns $M_*$, i.e.
$$
\mathcal B=\{\out=M_*\}.
$$
Since the probability of error of the algorithm is smaller than $\delta < 1/4$, we have $\Pr_0[\mathcal B] \ge 3/4$.
Define $\mathcal S$ be $\mathcal S=\mathcal A\cap \mathcal B \cap \mathcal C$. 
Then, by union bound, we have $\Pr_0[\mathcal S]\ge 1/4$.

Now, we show that if $\E_0[T_e] \le t_e^*$, then $\Pr_1[\mathcal B] \ge \delta$.
Let $W$ be the history of the sampling process until the algorithm stops (including the sequence of arms chosen at each time and the sequence of observed outcomes).
Define the likelihood function $L_l$ as 
$$
L_l(w) = p_l(W=w),
$$
where $p_l$ is the probability density function under hypothesis $H_l$.
Let $K$ be the shorthand of $K_e(T_e)$.

Assume that the event $\mathcal S$ occurred.
We will bound the likelihood ratio $L_1(W)/L_0(W)$ under this assumption. 
To do this, we divide our analysis into two different cases.

\textbf{Case (1): $e\not \in M_*$.}
In this case, the reward distribution of arm $e$ under $H_1$ is a Gaussian distribution with mean $p_e+2\Delta_e$ and variance 1. 
Recall that the probability density function of a Gaussian distribution with mean $\mu$ and variance $\sigma^2$ is given by
$\mathcal N(x | \mu,\sigma^2)=\frac{1}{\sigma\sqrt{2\pi}}\exp\left(-\frac{(x-\mu)^2}{2\sigma^2}\right)$.
Hence, we have
\begin{align}
  \frac{L_1(W)}{L_0(W)} &= \prod_{i=1}^{T_e} \exp\left(\frac{-(X_i-w(e)-2\Delta_e)^2+(X_i-w(e))^2}{2}\right) \nonumber \\
  						&= \prod_{i=1}^{T_e} \exp\big(\Delta_e(2X_i-2w(e))-2\Delta_e^2\big) \nonumber \\
  						&= \exp\big(\Delta_e(2K-2w(e)T_e)-2\Delta_e^2T_e\big) \nonumber \\
  						&= \exp\big(\Delta_e(2K-2w(e)T_e)\big)\exp(-2\Delta_e^2T_e) \label{eq:lower-bound-case-1}.
\end{align}

Next, we bound each individual term on the right-hand side of Eq.~\eqref{eq:lower-bound-case-1}.
We begin with bounding the second term of Eq.~\eqref{eq:lower-bound-case-1}
\begin{align}
	\exp(-2\Delta_e^2T_e) &\ge \exp(-8\Delta_e^2t_e^*) \label{eq:lower-bound-case-1-1.1} \\
						  &=\exp\left(-\frac{8}{16}\log(1/\theta)\right) \label{eq:lower-bound-case-1-1.2}\\
						  &= \theta^{1/2}\label{eq:lower-bound-case-1-1.3},
\end{align}
where Eq.~\eqref{eq:lower-bound-case-1-1.1} follows from the assumption that event $\mathcal S$ occurred, which implies that event $\mathcal A$ occurred and therefore $T_e \le 4t_e^*$; Eq.~\eqref{eq:lower-bound-case-1-1.2} follows from the definition of $t_e^*$.

Then, we bound the first term on the right-hand side of Eq.~\eqref{eq:lower-bound-case-1} as follows
\begin{align}
	\exp\big(\Delta_e(2K-2w(e)T_e)\big) & \ge \exp\left(-2\Delta_e\sqrt{t_e^*\log(1/\theta)}\right) \label{eq:lower-bound-case-1-2.1}\\
								       & = \exp\left(-\frac{2}{\sqrt{4}}\log(1/\theta)\right) \label{eq:lower-bound-case-1-2.2}\\
								       &=\theta^{1/2},  \label{eq:lower-bound-case-1-2.3}
\end{align}
where Eq.~\eqref{eq:lower-bound-case-1-2.1} follows from the assumption that event $\mathcal S$ occurred, which implies that event $\mathcal C$ and therefore $|2K-2w(e)T_e| \le \sqrt{t_e^*\log(1/\theta)}$; 
Eq.~\eqref{eq:lower-bound-case-1-2.2} follows from the definition of $t_e^*$.

Combining Eq.~\eqref{eq:lower-bound-case-1-1.3} and Eq.~\eqref{eq:lower-bound-case-1-2.3}, we can bound $L_1(W)/L_0(W)$ for this case as follows
\begin{equation}
\label{eq:ll-case1-final} 
\frac{L_1(W)}{L_0(W)} \ge \theta. 
\end{equation}


\emph{(End of Case (1).)}

\textbf{Case (2): $e\in M_*$.}
In this case, we know that the mean reward of arm $e$ under $H_1$ is $p_e-2\Delta$.
Therefore, the likelihood ratio $L_1(W)/L_0(W)$ is given by
\begin{align}
  \frac{L_1(W)}{L_0(W)} &= \prod_{i=1}^{T_e} \exp\left(\frac{-(X_i-w(e)+2\Delta_e)^2+(X_i-w(e))^2}{2}\right) \nonumber \\
  						&= \prod_{i=1}^{T_e} \exp\big(\Delta_e(2w(e)-2X_i)-2\Delta_e^2\big) \nonumber \\
  						&= \exp\big(\Delta_e(2w(e)T_e-2K)\big)\exp(-2\Delta_e^2T_e) \label{eq:lower-bound-case-2}.
\end{align}

Notice that the right-hand side of Eq.~\eqref{eq:lower-bound-case-2} differs from Eq.~\eqref{eq:lower-bound-case-1} only in its first term.
Now, we bound the first term as follows
\begin{align}
	\exp\big(\Delta_e(2K-2w(e)T_e)\big) & \ge \exp\left(-2\Delta_e\sqrt{t_e^*\log(1/\theta)}\right) \label{eq:lower-bound-case-2-2.1}\\
								       & = \exp\left(-\frac{2}{4}\log(1/\theta)\right) \label{eq:lower-bound-case-2-2.2}\\
								       &=\theta^{1/2},  \label{eq:lower-bound-case-2-2.3}
\end{align}
where the inequalities hold due to reasons similar to Case (1): Eq.~\eqref{eq:lower-bound-case-2-2.1} follows from the assumption that event $\mathcal S$ occurred, which implies that event $\mathcal C$ and therefore $|2K-2w(e)T_e| \le \sqrt{t_e^*\log(1/\theta)}$; 
Eq.~\eqref{eq:lower-bound-case-2-2.2} follows from the definition of $t_e^*$.

Combining Eq.~\eqref{eq:lower-bound-case-1-1.3} and Eq.~\eqref{eq:lower-bound-case-1-2.3}, we  can obtain the same bound of $L_1(W)/L_0(W)$ as in Eq.~\eqref{eq:ll-case1-final}, i.e. $L_1(W)/L_0(W) \ge \theta$.

\emph{(End of Case (2).)}

At this point, we have proved that, if the event $\mathcal S$ occurred, then the bound of likelihood ratio Eq.~\eqref{eq:ll-case1-final} holds, i.e. $\frac{L_1(W)}{L_0(W)} \ge \theta$.
Hence, we have
\begin{align}
\frac{L_1(W)}{L_0(W)} &\ge \theta \nonumber \\
					  &= 4\delta.	
\end{align}


Define $1_S$ as the indicator variable of event $\mathcal S$, i.e. $1_S = 1$ if and only if $\mathcal S$ occurs and otherwise $1_S = 0$.
Then, we have
\begin{align*}
\frac{L_1(W)}{L_0(W)} 1_S \ge 4\delta 1_S
\end{align*}
holds regardless the occurrence of event $\mathcal S$.
Therefore, we can obtain
\begin{align*}
\Pr_1[\mathcal B] &\ge \Pr_1[\mathcal S] = \E_1[1_S] \\
				  &= \E_0\left[\frac{L_1(W)}{L_0(W)} 1_S\right] \\
				  &\ge 4\delta \E_0[1_S] \\
				  &= 4\delta \Pr_0[\mathcal S] > \delta.
\end{align*}
Now we have proved that, if $\E_0[T_e] \le t_e^*$, then $\Pr_1[\mathcal B]>\delta$.
This means that, if $\E_0[T_e] \le t_e^*$, algorithm $\mathbb A$ will choose $M_*$ as the output with probability at least $\delta$, under hypothesis $H_1$.
However, under $H_1$, we have shown that $M_*$ is not the optimal set since $w_1(M_e) > w_1(M_*)$.
Therefore, algorithm $\mathbb A$ has a probability of error larger than $\delta$ under $H_1$. 
This contradicts to the assumption that algorithm $\mathbb A$ is a $\delta$-correct algorithm.
Hence, we must have $\E_0[T_e] > t_e^* = \frac{1}{16\Delta_e^2}\log(1/4\delta)$.

\end{proof}

\begin{proof}
Fix $\delta >0$, $\vec w\in \RR^{n}$, diff-set $b=(b_+,b_-)$ and a $\delta$-correct algorithm $\mathbb A$.
Assume that $\Rew_e(e)=\mathcal N(w(e),1)$ for all $e\in[n]$.

%the reward distribution of an arm $i\in [n]$ is a Gaussian distribution with mean $p_i$ and variance 1.


We define three hypotheses $H_0$, $H_1$ and $H_2$. 
%Under each of these hypotheses, the reward distribution of each arm is Gaussian with different means. 
Under hypothesis $H_0$, the reward distribution 
$$
H_0: \Rew_l = \mathcal N(w(l),1) \quad \text{for all } l \in [n].
$$
Under hypothesis $H_1$, the mean reward of each arm is given by 
$$
H_1: \Rew_e = \begin{cases}
	\mathcal N\left(w(e)+2\frac{w(b)}{|b_-|},1\right) & \text{if } e\in b_-,\\
	\mathcal N(w(e), 1) & \text{if } e\not\in b_-.\\
    \end{cases}
$$
And under hypothesis $H_2$, the mean reward of each arm is given by 
$$
H_2: \Rew_e = \begin{cases}
	\mathcal N\left(w(e)-2\frac{w(b)}{|b_-|},1\right)  & \text{if } e\in b_+,\\
	\mathcal N(w(e), 1)  & \text{if } e\not\in b_+.\\
    \end{cases}
$$

Since $b\in \Bopt$, it is clear that $\neg b \diffvalid M_*$. Hence we define $M = M_* \ominus b$.
Let $w_0, w_1$ and $w_2$ be the expected reward vectors under $H_0,H_1$ and $H_2$ respectively.
It is easy to check that 
$w_1(M_*)-w_1(M) = -w(b) < 0$ and
$w_2(M_*)-w_2(M) = -w(b) < 0$.
This means that under $H_1$ or $H_2$, $M_*$ is not the optimal set.
Further, for $l\in \{0,1,2\}$, we use $\E_l$ and $\Pr_l$ to denote the expectation and probability, respectively, under the hypothesis $H_l$.
In addition, let $W$ be the history of the sampling process until algorithm $\mathbb A$ stops.
Define the likelihood function $L_l$ as 
$$
L_l(w) = p_l(W=w),
$$
where $p_l$ is the probability density function under $H_l$.


Define $\theta=4\delta$.
Let $T_{b_-}$ and $T_{b_+}$ denote the number of trials of arms belonging to $b_-$ and $b_+$, respectively. 
In the rest of the proof, we will bound $\E_0[T_{b_-}]$ and $\E_0[T_{b_+}]$ individually.



\textbf{Part (1): Lower bound of $\E_0[T_{b_-}]$.}
In this part, we will show that $\E_0[T_{b_-}]\ge t_{b_-}^*$, where we define $t_{b_-}^* = \frac{|b_-|^2}{16 w(b)^2}\log(1/\theta)$.

Consider the complete sequence of sampling process by algorithm $\mathbb A$.
Formally, let $W=\{(\tilde I_1,\tilde X_1),\ldots, (\tilde I_T, \tilde X_T)\}$ be the sequence of all trials by algorithm $\mathbb A$, where $\tilde I_i$ denotes the arm played in $i$-th trial and $\tilde X_i$ be the reward outcome of $i$-th trial.
Then, consider the subsequence $W_1$ of $W$ which consists all the trials of arms in $b_-$.
Specifically, we write $W=\{(I_1,X_1),\ldots,(I_{T_{b_-}}, X_{T_{b_-}})\}$ such that $W_1$ is a subsequence of $W$ and $I_i \in b_-$ for all $i$.

Next, we define several random events in a way similar to the proof of Theorem~\ref{theorem:lower-bound}.
Define event
$\mathcal A_1 = \{T_{b_-} \le 4t_{b_-}^* \}$.
Define event 
$$
\mathcal C_1 = \left\{\max_{1\le t \le 4t_{b_-}^*} \left|\sum_{i=1}^t X_i - \sum_{i=1}^t w(I_i)\right|  < \sqrt{t_{b_-}^*\log(1/\theta)} \right\}.
$$
Define event 
\begin{equation}
\label{eq:lower-sum-b-define}
\mathcal B = \{\out=M_*\}.
\end{equation}
Define event
$\mathcal S_1 = \mathcal A_1 \cap \mathcal B \cap \mathcal C_1$.
Then, we bound the probability of events $\mathcal A_1$, $\mathcal B$, $\mathcal C_1$ and $\mathcal S_1$ under $H_0$ using methods similar to Theorem~\ref{theorem:lower-bound}.
First, we show that $\Pr_0[\mathcal A_1] \ge 3/4$. 
This can be proved by Markov inequality as follows.
\begin{align*}
\Pr_0[T_{b_-} > 4t_{b_-}^*] &\le \frac{\E_0[T_{b_-}]}{4t_{b_-}^*} \\
					  &= \frac{t_{b_-}^*}{4t_{b_-}^*} = \frac14.
\end{align*}
Next, we show that $\Pr_0[\mathcal C_1] \ge 3/4$.
Notice that the sequence $\Big\{\sum_{i=1}^t X_i - \sum_{i=1}^t p_{I_i}\Big\}_{t\in[4t_{b_-}^*]}$ is a martingale.
Hence, by Kolmogorov's inequality, we have
\begin{align*}
\Pr_0\left[\max_{1\le t \le 4t_{b_-}^*} \left|\sum_{i=1}^t X_i - \sum_{i=1}^t w(I_i) \right| \ge \sqrt{t_e^*\log(1/\theta)} \right]
&\le \frac{\E_0\left[ \left(\sum_{i=1}^{4t_{b_-}^*} X_i - \sum_{i=1}^{4t_{b_-}^*} w(I_i)\right)^2\right]}{t_e^*\log(1/\theta)}\\
&= \frac{4t_{b_-}^*}{t_{b_-}^*\log(1/\theta)}\\
&< \frac14,
\end{align*}
where the second inequality follows from the fact that all reward distributions have unit variance and hence
$\E_0\left[ \left(\sum_{i=1}^{4t_{b_-}^*} X_i - \sum_{i=1}^{4t_{b_-}^*} p_{I_i}\right)^2\right] = 4t_{b_-}^*$; the last inequality follows 
since $\theta < e^{-16}$.
Last, since algorithm $\mathbb A$ is a $\delta$-correct algorithm with $\delta < 1/4$. 
Therefore, it is easy to see that 
$\Pr_0[\mathcal B] \ge 3/4$.
And by union bound, we have
$$
\Pr_0[\mathcal S_1] \ge 1/4.
$$

Now, we show that if $\E_0[T_{b_-}] \le t_{b_-}^*$, then $\Pr_1[\mathcal B] \ge \delta$.
Assume that the event $\mathcal S_1$ occurred.
%Define $K(e) = \sum_{t=1}^{T_e} $ 
We bound the likelihood ratio $L_1(W)/L_0(W)$ under this assumption as follows
\begin{align}
  \frac{L_1(W)}{L_0(W)} 
  &= \prod_{i=1}^{T_{b_-}}
  \exp\left(\frac{-\left(X_i-w(I_i)-\frac{2w(b)}{|b_-|}\right)^2+(X_i-w(I_i)^2}{2}\right) \nonumber \\
  &= \prod_{i=1}^{T_{b_-}}
  \exp\left(\frac{w(b)}{|b_-|}(2X_i-2w(I_i))-\frac{2w(b)^2}{|b_-|^2}\right) \nonumber \\
  &= \exp\left(\frac{w(b)}{|b_-|}\left(\sum_{i=1}^{T_{b_-}}2X_i-2w(I_i)\right)-\frac{2w(b)^2}{|b_-|^2}T_{b_-}\right) \nonumber \\
  &= \exp\left(\frac{w(b)}{|b_-|}\left(\sum_{i=1}^{T_{b_-}}2X_i-2w(I_i)\right)\right)\exp\left(-\frac{2w(b)^2}{|b_-|^2}T_{b_-}\right) \label{eq:lower-sum-case-1}.
\end{align}
Then, we bound each term on the right-hand side of Eq.~\eqref{eq:lower-sum-case-1}.
First, we bound the second term of Eq.~\eqref{eq:lower-sum-case-1}.
\begin{align}
	\exp\left(-\frac{2w(b)^2}{|b_-|^2} T_{b_-}\right) 
	 &\ge \exp\left(-\frac{2w(b)^2}{|b_-|^2} 4t_b^*\right) \label{eq:lower-sum-case-1-1.1} \\
     &=\exp\left(-\frac{8}{16}\log(1/\theta)\right) \label{eq:lower-sum-case-1-1.2}\\
     &= \theta^{1/2}\label{eq:lower-sum-case-1-1.3},
\end{align}
where Eq.~\eqref{eq:lower-sum-case-1-1.1} follows from the assumption that events $\mathcal S_1$ and $\mathcal A_1$ occurred and therefore $T_{b_-} \le 4t_{b_-}^*$; 
Eq.~\eqref{eq:lower-sum-case-1-1.2} follows from the definition of $t_{b_-}^*$.
Next, we bound the first term of Eq.~\eqref{eq:lower-sum-case-1} as follows
\begin{align}
	\exp\left(\frac{w(b)}{|b_-|}\left(\sum_{i=1}^{T_{b_-}}2X_i-2w(I_i)\right)\right)
	&\ge \exp\left(-\frac{2w(b)}{|b_-|}\sqrt{t_b^*\log(1/\theta)}\right) \label{eq:lower-sum-case-1-2.1}\\
    & = \exp\left(-\frac{2}{4}\log(1/\theta)\right) \label{eq:lower-sum-case-1-2.2}\\
    &=\theta^{1/2},  \label{eq:lower-sum-case-1-2.3}
\end{align}
where Eq.~\eqref{eq:lower-sum-case-1-2.1} follows since event $\mathcal S_1$ and $\mathcal C_1$ occurred and therefore $|2K-2p_eT_e| \le \sqrt{t_e^*\log(1/\theta)}$; 
Eq.~\eqref{eq:lower-sum-case-1-2.2} follows from the definition of $t_{b_-}^*$.

Hence, if event $\mathcal S_1$ occurred, we can bound the likelihood ratio as follows
\begin{align}
\frac{L_1(W)}{L_0(W)} &\ge \theta = 4\delta.
\end{align}
Let $1_{S_1}$ denote the indicator variable of event $\mathcal S_1$.
Then, we have $
\frac{L_1(W)}{L_0(W)} 1_{S_1} \ge 4\delta 1_{S_1}$.
Therefore, we can bound $\Pr_1[\mathcal B]$ as follows
\begin{align}
\Pr_1[\mathcal B] &\ge \Pr_1[\mathcal S_1] = \E_1[1_{S_1}] \nonumber \\
				  &= \E_0\left[\frac{L_1(W)}{L_0(W)} 1_{S_1}\right] \nonumber \\
				  &\ge 4\delta \E_0[1_{S_1}] \nonumber \\
				  &= 4\delta \Pr_0[\mathcal S_1] > \delta \label{eq:lower-sum-case-1-final}.
\end{align}
This means that, if $\E_0[T_{b_-}] \le t_{b_-}^*$, then, under $H_1$, the probability of algorithm $\mathbb A$ returning $M_*$ as output is at least $\delta$. 
But $M_*$ is not the optimal set under $H_1$. Hence this contradicts to the assumption that $\mathbb A$ is a $\delta$-correct algorithm.
Hence we have proved that 
\begin{equation}
\label{eq:lower-sum-case-1-a}
\E_0[T_{b_-}] \ge t_{b_-}^* = \frac{|b_-|^2}{16 w(b)^2}\log(1/4\delta).
\end{equation}

\emph{(End of Part (1).)}

\textbf{Part (2): Lower bound of $\E_0[T_{b_+}]$.} In this part, we will show that $\E_0[T_{b_+}]\ge t_{b_+}^*$, where we define $t_{b_+}^* = \frac{|b_+|^2}{16 w(b)^2}\log(1/\theta)$.
The arguments used in this part are similar to that of Part (1). 
Hence, we will omit the redundant parts and highlight the differences.

Recall that we have defined that $W$ to be the history of all trials by algorithm $\mathbb A$.
We define $W$ be the subsequence of $\tilde S$ which contains the trials of arms belonging to $b_+$.
We write $S_2=\{(J_1,Y_1),\ldots,(J_{T_{b_+}}, Y_{T_{b_+}})\}$, where $J_i$ is $i$-th played arm in sequence $S_2$ and $Y_i$ is the associated reward outcome.

We define the random events $\mathcal A_2$ and $\mathcal C_2$ similar to Part (1).
Specifically, we define 
$$
\mathcal A_2 = \{T_{b_+} \le 4t_{b_+}^* \} \quad\text{and}\quad
\mathcal C_2 = \left\{\max_{1\le t \le 4t_{b_+}^*} \left|\sum_{i=1}^t Y_i - \sum_{i=1}^t w(J_i)\right|  < \sqrt{t_{b_+}^*\log(1/\theta)} \right\}.
$$
Using the similar arguments, we can show that 
$\Pr_0[\mathcal A_2] \ge 3/4$ and $\Pr_0[\mathcal C_2] \ge 3/4$.
Define event $\mathcal S_2 = \mathcal A_2 \cap \mathcal B \cap \mathcal C_2$, where $\mathcal B$ is defined in Eq.~\eqref{eq:lower-sum-b-define}.
By union bound, we see that 
$$
\Pr_0[\mathcal S_2] \ge 1/4.
$$

Then, we show that if $\E_0[T_{b_+}] \le t_{b_+}^*$, then $\Pr_2[\mathcal B] \ge \delta$.
We bound likelihood ratio $L_2(W)/L_0(W)$ under the assumption that $\mathcal S_2$ occurred as follows
\begin{align}
  \frac{L_2(W)}{L_0(W)} 
  &= \prod_{i=1}^{T_{b_+}}
  \exp\left(\frac{-\left(Y_i-w(J_i))+\frac{2w(b)}{|b_-|}\right)^2+(Y_i-w(J_i))^2}{2}\right) \nonumber \\
  &= \prod_{i=1}^{T_{b_+}}
  \exp\left(\frac{w(b)}{|b_+|}(2w(J_i)- 2Y_i)-\frac{2w(b)^2}{|b_+|^2}\right) \nonumber \\
  &= \exp\left(\frac{w(b)}{|b_+|}\left(\sum_{i=1}^{T_{b_+}}2w(J_i)-2Y_i\right)-\frac{2w(b)^2}{|b_+|^2}T_{b_+}\right) \nonumber \\
  &= \exp\left(\frac{w(b)}{|b_+|}\left(\sum_{i=1}^{T_{b_+}}2w(J_i)-2Y_i\right)\right)\exp\left(-\frac{2w(b)^2}{|b_+|^2}T_{b_+}\right) \nonumber \\
  &\ge \theta \label{eq:lower-sum-case-2}\\
  & = 4\delta \nonumber,
\end{align}
where Eq.~\eqref{eq:lower-sum-case-2} can be obtained using same method as in Part (1) as well as the assumption that $\mathcal S_2$ occurred.

Next, similar to the derivation in Eq.~\eqref{eq:lower-sum-case-1-final}, we see that
$$
\Pr_2[\mathcal B] \ge \Pr_2[\mathcal S_2] = \E_2[1_{S_2}] 
				  = \E_0\left[\frac{L_2(W)}{L_0(W)} 1_{S_2}\right]
				  \ge 4\delta \E_0[1_{S_2}] > \delta,
$$
where $1_{S_2}$ is the indicator variable of event $\mathcal S_2$.
Therefore, we see that if $\E_0[T_{b_+}] \le t_{b_+}^*$, then, under $H_2$, the probability of algorithm $\mathbb A$ returning $M_*$ as output is at least $\delta$, which is not the optimal set under $H_2$. 
This contradicts to the assumption that algorithm $\mathbb A$ is a $\delta$-correct algorithm. 
In sum, we have proved that 
\begin{equation}
\label{eq:lower-sum-case-2-a}
\E_0[T_{b_+}] \ge t_{b_+}^* = \frac{|b_+|^2}{16 w(b)^2}\log(1/4\delta).
\end{equation}

\emph{(End of Part (2))}

Finally, we combine the results from both parts, i.e. Eq.~\eqref{eq:lower-sum-case-1-a} and Eq.~\eqref{eq:lower-sum-case-2-a}.
We obtain
\begin{align*}
  \E_0[T_b] &= \E_0[T_{b_-}]+\E_0[T_{b_+}] \\
  		    &\ge \frac{|b_+|^2+|b_-|^2}{16 w(b)^2}\log(1/4\delta) \\
  		    &\ge \frac{|b|^2}{32 w(b)^2}\log(1/4\delta).
\end{align*}
\end{proof}

Now we prove a lower bound on the probability of error in the fixed budget setting.
In particular, we show that for any expected rewards $\{w(1),\ldots, w(n)\}$ and any fixed budget algorithm $\mathbb A$ for pure exploration combinatorial bandit problem with feasible sets $\M$. 
We show that one can slightly modify the vector $\{w(e)\}_{e\in [n]}$ to construct another vector $\{\tilde w(1),\ldots,\tilde w(n)\}$ such that the probability of error of the fixed algorithm on a \Problem problem with expected rewards given by $\tilde w$ is at least $\Omega(\exp(n/\mathbf H(w)))$.
%In addition, newthe modified vector $\{\tilde w(1)\}$
%On the other hand, it is clear that one cannot prove any non-trivial lower bounds on the probability of error of algorithm $\mathbb A$: an algorithm which always outputs $M_*$ has zero probability of error for this particular problem instance.
%We also notice that Theorem~\ref{theorem:lower-bound} does not imply 



\section{Technical Lemmas}

\label{section:technical}

\begin{lemma}[Basis exchange property]
AA
\label{lemma:basis-exchange-matroid}
\end{lemma}

\begin{lemma}[Hoeffding's inequality]
\label{lemma:hoeffeding}
Let $X_1,\ldots, X_n$ be $n$ independent $R$-sub-Gaussian random variables. 
Let $\bar X=\frac{1}{n}\sum X_i$ be the average of these random variables.
Then, we have
$$
\Pr\Big[ \big|\bar X - \mathbb E[\bar X] \big| \ge t \Big] \le 
2\exp\left(-\frac{2nt^2}{R^2}\right).
$$
\end{lemma}

\section{Trash}




\begin{define}[Optimal diff-sets]
Given a diff-set class $\B$ and the optimal set $M_*$.
We define $\Bopt$ as a subset of $\B$, and for all $b\in \B$, $b\in \Bopt$ if and only if,
there exists $M\not=M_*$ and $M_* \ominus M$ can be decomposed as $b,b_1,\ldots,b_k$ on $\B$.
\end{define}

\begin{define}[Hardness $\Delta_e$ of base arm $e$]
For each $e\in [n]$, we define its hardness $\Delta_e$ as follows
$$
\Delta_e = \min_{b\in \Bopt, e\in b} \frac{1}{\rank(\B)} w(b).
$$
\end{define}

\begin{define}[Sufficient exploration]
For all $t>0$, we define $E_t^3 \subseteq [n]$, such that, for all $e\in[n]$
$e\in E_t^3$ if and only if $\rad_t(e) < \frac{1}{3} \Delta_e$.
\end{define}

\begin{corollary}
For all $t>0$ and $e\in[n]$
$$ n_t(e) \ge O(\frac{1}{\Delta_e^2}\log(\Delta_e n/\delta)) \implies e\in E_t^3.$$
\end{corollary}

\begin{theorem}
With probability at least $1-\delta$,
the algorithm returns $M_*$,
and the number of samples used by the algorithm are at most
$$
\sum_{e\in [n]} \Delta_e^{-2}\log(\Delta_e n/\delta).
$$
\end{theorem}


\begin{theorem}
Given confidence parameter $\delta \in (0,1)$, tolerance parameter $\epsilon \ge 0$, number of arms $n$ and a combinatorial problem instance $\mathcal M \subseteq 2^{[n]}$.
Let oracle $\Oracle(w)$ be a maximization oracle associated with $\mathcal M$ such that
$\Oracle(w) = \argmax_{M \in \mathcal M} w(M)$, where $w: 2^{[n]} \rightarrow R$ is a weight function.

Then, with probability at least $1-\delta$, the output $\out$ of Algorithm~\ref{algo:pac} satisfies
$
w(M_*)-w(\out) \le \epsilon,
$
where $M_* = \argmax_{M\in \M} w(M)$ is the optimal set.
In addition, the number of samples $T$ used by the algorithm satisfies
$$
T \le \mathbf H_\epsilon \log\left(\frac{n}{\delta}\mathbf H_\epsilon\right),
$$
where
$$
\mathbf H_\epsilon = \sum_{e\in[n]} \min\left\{\frac{\rank(\B)^2}{\Delta_e^2}, \frac{n^2}{\epsilon^2}\right\}.
$$
\end{theorem}

\begin{lemma}
For any arm $e \in [n]$ and any round $t > n$ after initialization, if $\rad_t(e) \le \max\left\{\frac{\Delta_e}{3\rank(\B)}, \frac{\epsilon}{n}\right\}$,
then arm $e$ will not be played on round $t$, i.e. $p_t\not= e$.
\end{lemma}
\begin{proof}
If $\rad_t(e) \le \frac{\Delta_e}{3\rank(\B)}$, then we can apply Lemma~\ref{lemma:key-technical} which immediately gives that $p_t\not= e$.
Hence, we only need to prove the case that $\rad_t(e) \le \frac{\epsilon}{n}$.
By the definition of $p_t$, we know that for each $i\in D_t$, we have $\rad_t(i) \le \rad_t(e) \le \frac{\epsilon}{n}$.
Summing up all $i\in D_t$, we obtain
\begin{equation}
\rad_t(D_t) \le \epsilon.
\label{eq:pac-key-0-1}
\end{equation}
Next, we notice that the definition of $M_t$ gives that $\bar w_t(M_t)=\max_{M\in \M} \bar w_t(M) \ge \bar w_t(M_t \oplus D_t)$.
This means that
\begin{equation}
\bar w_t(D_t) = \bar w_t(M_t\oplus D_t)-\bar w_t(M_t) \le 0.
\label{eq:pac-key-0-2}
\end{equation}
Using the above inequalities, we have.
\begin{align}
  w_t^+(D_t) &= \bar w_t(D_t)+\rad_t(D_t) \label{eq:pac-key-1}\\
  			 &\le \bar w_t(D_t)+\epsilon \label{eq:pac-key-2}\\
  			 &\le \epsilon, \label{eq:pac-key-3}
\end{align}
where Eq.~\eqref{eq:pac-key-1} follows from the definition of $w_t^+(\cdot)$; 
Eq.~\eqref{eq:pac-key-2} follows from Eq.~\eqref{eq:pac-key-0-1};
Eq.~\eqref{eq:pac-key-3} holds since Eq.~\eqref{eq:pac-key-0-2}.
\end{proof}

\section{Preliminaries}

\subsection{Problems}

Let $n$ be the number of base arms. 
Let $\M \subseteq 2^{[n]}$ be the set of super arms. 


In this note, we consider the following cases of $\M$.

\begin{example}[Explore-$m$]
$\mtop(n)=\{M \subseteq [n] \;|\; |M|=m\}$.
This corresponds to finding the top $m$ arms from $[n]$.
\end{example}

\begin{example}[Explore-$m$-bandits]
Suppose $n=mk$. Then $\mbandit(n)$ contains all subsets $M \subseteq [n]$ with size $m$, such that 
$$ 
M\cap \{ik+1,\ldots, (i+1)k\} = 1, \quad \text{for all } i \in \{0,\ldots, m-1\}.
$$ 
This corresponds to finding the top arms from $m$ bandits, where each bandit has $k$ arms.
\end{example}

\begin{example}[Perfect Matching]
Let $G=(V,E)$ be a bipartite graph and $|E|=n$. 
For simplicity, let each edge $e\in E$ corresponds to a unique integer $i\in [n]$, and vice versa. 
Then $\mmatch(n,G)$ contains all subsets $M \subseteq [n]$ such that $M$ corresponds to a perfect matching in $G$.
\end{example}


\subsection{Diff-Sets}

\begin{define}[Diff-set]
An $n$-diff-set (or diff-set in short) is a pair of sets $c=(c_+,c_-)$, where $c_+\subseteq[n]$, $c_-\subseteq [n]$ and $c_+\cap c_-=\emptyset$.
\end{define}


\begin{define}[Difference of sets]
Given any $M_1\subseteq[n],M_2\subseteq[n]$. We define $M_1\ominus M_2 \triangleq C$, where $C=(C_+,C_-)$ is a diff-set and
$C_+ = M_1 \del M_2$ and $C_- = M_2\del M_1$.
\end{define}

\begin{define}
Denote $\diff[n]$ be the set of all possible $n$-diff-sets.
\end{define}

\begin{define}[Set operations of diff-sets] 
Let $C=(C_+,C_-), D=(D_+,D_-)$ be two diff-sets. 
We define 
$C\cap D \triangleq (C_+\cap D_+, C_-\cap D_-)$
and $C\del D \triangleq (C_+\del D_+, C_-\del D_-)$.

Further, for all $e\in [n]$, $e \in C \Leftrightarrow (e\in C_+)\vee(e\in C_-)$.
And $|C|\triangleq |C_+|+|C_-|$.
\end{define}

\begin{define}[Valid diff-set]
Given a set $M \subseteq [n]$ and a diff-set $C=(C_+,C_-)$, we call $C$ a \emph{valid diff-set} for $M$, iff $C_+ \cap M = \emptyset$ and $C_- \subseteq M$.
In this case, we denote $C\diffvalid M$.
\end{define}

\begin{define}[Negative diff-set]
Given a diff-set $A=(A_+,A_-)$, we define $\neg A=(A_-,A_+)$.
\end{define}


\subsubsection{diff-set operations}

\begin{define}[Operators $\oplus$ and $\ominus$]
Given any $M \subseteq [n]$ and $C \in \diff[n]$.
If $C\diffvalid M$, we define operator $\oplus$ such that $M \oplus C \triangleq M\del C_- \cup C_+ $.
On the other hand if $\neg C\diffvalid M$, we define operator $\ominus$ such that $M \ominus C \triangleq M\oplus (\neg C) = 
M\del C_+ \cup C_- $.
\end{define}


\begin{define}
Given two diff-sets $A=(A_+,A_-)$ and $B=(B_+,B_-)$.
We denote $B \diffvalid A$, if and only if $B_+\cap A_+ = \emptyset$ and $A_+\cap A_-=\emptyset$.
\end{define}

\begin{define}
Given two diff-sets $A=(A_+,A_-)$ and $B=(B_+,B_-)$. 
If $B\diffvalid A$, we define $A\oplus B = ( (A_+\cup B_+)\del(A_-\cup B_-), (A_-\cup B_-)\del(A_+\cup B_+))$.
\end{define}

\begin{lemma}
Given two diff-sets $A=(A_+,A_-)$ and $B=(B_+,B_-)$. 
If $B\diffvalid A$, then 
$A\oplus B$ is a diff-set.
\end{lemma}

\begin{proof}
Let $C=A\oplus B$.
By definition, we have $C_+ = (A_+\cup B_+)\del(A_-\cup B_-)$ and $C_-=(A_-\cup B_-)\del(A_+\cup B_+)$.

We only need to show that $C_+\cap C_-=\emptyset$.
\begin{align*}
	C_+ \cap C_- &= \big((A_+\cup B_+)\del(A_-\cup B_-)\big)\cap\big((A_-\cup B_-)\del(A_+\cup B_+)\big)\\
			     &= (A_+\cup B_+)\cap\big((A_-\cup B_-)\del(A_+\cup B_+) \del (A_-\cup B_-)\big) \\
			     &= \emptyset.
\end{align*}
\end{proof}

\begin{lemma}
\label{lemma:diff-set-algebra}
Given two diff-sets $A=(A_+,A_-)$ and $B=(B_+,B_-)$. 
If there exists $M\subseteq [n]$ such that $A \diffvalid M$, and $B \diffvalid (M\oplus A)$,
then $B \diffvalid A$ and $(M\oplus A \oplus B)\ominus M = A\oplus B$.
\end{lemma}

\begin{proof}
We first show that $B\diffvalid A$.
Since $B \diffvalid (M\oplus A)$, we know that 
$B_+\cap (M\del A_- \cup A_+) = \emptyset$.
Therefore, we have
\begin{align*}
	\emptyset &= B_+\cap (M\del A_-\cup A_+) \\
			  &= (B_+\cap (M\del A_-))\cup (B_+\cap A_+)
\end{align*}
We see that $B_+\cap A_+=\emptyset$.

On the other hand, we have $B_-\subseteq (M\del A_-\cup A_+)$, therefore 
\begin{align*}
	B_-\cap A_- &\subseteq (M\del A_-\cup A_+)\cap A_- \\
				&= (M\del A_- \cap A_-)\cup (A_+\cap A_-) \\
				&= \emptyset.
\end{align*}
Hence we proved that $B\diffvalid A$.

Define $D=(M\oplus A \oplus B)\ominus M$ and write $D=(D_+,D_-)$. Then,
\begin{align*}
D_+ &= (M\oplus A\oplus B) \del M\\ 
    &= (M\del A_-\cup A_+\del B_-\cup B_+) \del M\\
    &= (A_+\cup B_+)\del(A_-\cup B_-).
\end{align*}
Similarly, we have
\begin{align*}
D_- &= M\del (M\oplus A\oplus B)\\ 
    &= M\del (M\del A_-\cup A_+\del B_-\cup B_+)\\
    &= (A_-\cup B_-)\del (A_+\cup B_+).
\end{align*}
\end{proof}

\subsubsection{Diff-set class}

\begin{define}[Decomposition of diff-set]
Given $\B\subseteq \diff[n]$ and $D\in \diff[n]$, 
a decomposition of $D$ on $\B$ is a set $\{b_1,\ldots,b_k\} \subseteq \B$ satisfying the following
\begin{enumerate}
  \item For all $i\in[k]$ and $j\in [k]$, we write $b_i=(b_i^+,b_i^-)$ and $b_j=(b_j^+,b_j^-)$. Then, the following holds
  $b_i^+\cap b_j^+=\emptyset$, $b_i^+\cap b_j^-=\emptyset$, $b_i^-\cap b_j^+ =\emptyset$ and $b_i^-\cap b_j^-=\emptyset$.
  \item $D=b_1 \oplus b_2 \oplus \ldots b_k$.
\end{enumerate}
\end{define}

\begin{lemma}
Given $\B\subseteq \diff[n]$ and $D\in \diff[n]$.
Let $\{b_1,\ldots,b_k\} \subseteq \B$ be a decomposition of $D$ on $\B$.
Then,
\begin{enumerate}
\item Let $D=(D_+,D_-)$ and for all $i \in [k]$, we write $b_i = (b_i^+,b_i^-)$.
	  Then $D_+=b_1^+ \cup \ldots \cup b_k^+$ and $D_-=b_1^-\cup\ldots\cup b_k^-$.
\item For all $M\subseteq [n]$, if $D\diffvalid M$, then, for all $i\in [k]$, we have $b_i \diffvalid M$.
\end{enumerate}
\end{lemma}

\begin{proof}
We prove (1) by induction.
Let $D_i = b_1 \oplus \ldots \oplus b_i$ and write $D_i=(D_i^+, D_i^-)$.
We show that $D_i^+=\bigcup_{j=1}^i b_i^+$ and $D_{i-}=\bigcup_{j=1}^i b_i^-$ for all $i\in[k]$.
For $i=1$, this is trivially true.
Then, assume that this is true for some $i>1$.
By definition $D_{i+1}=D_{i}\oplus b_{i+1}$, hence $D_{i+1}^+=(D_i^+ \cup b_{i+1}^+)\del(D_i^- \cup b_{i+1}^-)$.
Note that 
\begin{align*}
(D_i^-\cup b_{i+1}^-)\cap(D_i^+ \cup b_{i+1}^+) &= (D_i^-\cap D_i^+)\cup(D_i^- \cap b_{i+1}^+)\cup(b_{i+1}^- \cap D_i^+)\cup(b_{i+1}^- \cap b_{i+1}^+)\\
		&= (D_i^- \cap b_{i+1}^+) \cup (b_{i+1}^- \cap D_i^+) \\
		&= \left(\left(\bigcup_{j=1}^i b_{j}^-\right) \cap b_{i+1}^+ \right) \cup \left(\left(\bigcup_{j=1}^i b_{j}^+\right) \cap b_{i+1}^- \right)\\
		&= \emptyset.
\end{align*}
Hence $D_{i+1}^+=D_i^+ \cup b_{i+1}^+$. We can use the same method to show that $D_{i+1}^-=D_i^- \cup b_{i+1}^-$.

Next, we prove (2) using (1).
To show that $b_i\diffvalid M$, we only need to show that $b_i^+ \cap M = \emptyset$ and $b_i^- \subseteq M$.
Since $D\diffvalid M$, we know that $D_+\cap M=\emptyset$ and $D_-\subseteq M$.
By (1), we see that $b_i^+\subseteq D_+$ and $b_i^-\subseteq D_-$.
Therefore, we have $(b_i^+\cap M) \subseteq (D_+\cap M) = \emptyset$ and $b_i^-\subseteq D_- \subseteq M$.

\end{proof}


\begin{define}[diff-set class]
\label{define:diff-class}
Given $\M \subseteq 2^{[n]}$. $\B \subseteq \diff[n]$ is a diff-set class for $\M$, if the following hold.
\begin{enumerate} 
\item $(\emptyset,\emptyset)\not\in \B$.
\item For all $M\in \M$ and for all $b\in \B$, if $b\diffvalid M$, then $M\oplus b \in \M$.
\item For all $M_1 \in \M$ and $M_2\in \M$, where $M_1\not=M_2$. 	
	  Let $D=M_1\ominus M_2$.  
	  Then, there exists a decomposition of $D$ on $\B$.
\end{enumerate}
\end{define}

\begin{define}[Rank of diff-set class]
Let $\B \subseteq [n]$ be a diff-set class for some $\M$. We define
$$
\rank(\B) \triangleq \max_{b \in \B} |b|.
$$
\end{define}

\begin{example}[diff-set class for Explore-$m$]
One diff-set class $\B$ for $\mtop(n)$ is given by
$$
\B=\{(\{b_1\},\{b_2\}) \;|\; b_1\not=b_2, b_1\in[n], b_2\in[n]\}.
$$
Proof omitted.
Further, we see that $\rank(\B)=2$.
\end{example}

\begin{example}[diff-set class for Explore-$m$-badit]
Let $n=mk$.
One diff-set class $\B$ for $\mbandit(n)$ is given by
$$
\B=\{(\{b_1\},\{b_2\}) \;|\; b_1\not=b_2, \exists i\in\{0,\ldots,k-1\}, b_1\in \{ik+1,\ldots, (i+1)k\}, b_2\in\{ik+1,\ldots, (i+1)k\}\}.
$$
Proof omitted.
Further, we see that $\rank(\B)=2$.
\end{example}


\begin{example}[diff-set class for Perfect Matching]
One diff-set class $\B$ for $\mmatch(n, G)$ is the set of all augmenting cycles of $G$. 
More specifically,
$$
\B=\{(b_+,b_-) | b_+\cup b_- \text{ is a cycle of } G\}.
$$

Note $\rank(\B)\le n$.
\end{example}

\subsection{Weights and confidence bounds}

\begin{define}[Weight functions]
Define function $w: [n] \rightarrow \RR^+$ which represents the weight of each base arm. 
Further, we slight abuse the notations, and extend the definition of $w$ to diff-sets and sets as follows.
\begin{enumerate}
\item For all $M \subseteq [n]$, we denote $w(M) = \sum_{e\in M} w(e)$.
\item For all $b=(b_+,b_-) \in \diff[n]$, we denote $w(b) = \sum_{e\in b_+} w(e)-\sum_{e\in b_-}w(e)$.
\end{enumerate}
\end{define}


\begin{lemma}
\label{lemma:weight-diff-simple}
Let $c\in \diff[n],d\in \diff[n]$. Let $w$ be a weight function.
Then,
\begin{equation}
w(c\cup d) = w(c)+w(d)-w(c\cap d).
\end{equation}
\end{lemma}

\begin{proof}
Let $c=(c_+,c_-)$ and $d=(d_+,d_-)$.
We have
\begin{align}
w(c\cup d) &= w(c_+\cup d_+)-w(c_-\cup d_-)\\
           &= w(c_+)+w(d_+)-w(c_+\cap d_+)-w(c_-)-w(d_-)+w(c_- \cap d_-)\\
           &= w(c)+w(d)-(w(c_+\cap d_+)-w(c_-\cap d_-))\\
           &= w(c)+w(d)-w(c\cap d).
\end{align}
\end{proof}

\begin{define}[Mean weight $\bar w_t$, sample size $n_t$]
Given $t>0$. 
Define $\bar w_t$ be a weight function such that, for all $e\in[n]$, $\bar w_t(e)$ equals to the empirical mean of $e$ up to round $t$.
Let $n_t: [n] \rightarrow \mathbb N$, such that $n_t(e)$ equals to number of plays of base arm $e$ up to round $t$.
\end{define}

\begin{define}[Confidence radius $\rad_t$]
Given $n$ and $t>0$.
Define $\rad_t:[n]\rightarrow \RR^+$ satisfying, for all $e\in[n]$,
\begin{equation}
\label{eq:define-confidence-radius}
\rad_t(e) = c_{\rad}\log\left(\frac{c_\delta nt^2}\delta\right)\frac{1}{\sqrt{n_t(e)}},
\end{equation}
where $c_{\rad} > 0$  and $c_\delta>0$ are some universal constants (specify later) and $\delta > 0$ is a parameter.

We extend the notation of $\rad_t$ to diff-sets and sets as follows.
\begin{enumerate}
\item For all $M \subseteq [n]$, $\rad_t(M) \triangleq \sum_{e\in M} \rad_t(e)$.
\item For all $b=(b_+,b_-)\in \diff[n]$, $\rad_t(b) \triangleq \rad_t(b_+)+\rad_t(b_-)$.
\end{enumerate}

\end{define}

\begin{define}[UCB $w_t^+$]
Define $w^+_t: [n] \rightarrow \RR^+$, s.t., for all $e\in[n]$,  
$$ w^+_t(e)=\bar w_t(e)+\rad_t(e).$$

We extend the notation of $w_t^+$ to diff-sets and sets as follows.
\begin{enumerate}
\item For all $M \subseteq [n]$, $w_t^+(M) \triangleq \bar w_t(M)+\rad_t(M)$.
\item For all $b=(b_+,b_-)\in \diff[n]$, $w_t^+(b) \triangleq \bar w_t(b)+\rad_t(b)$.
\end{enumerate}

\end{define}

\begin{lemma}
\label{lemma:conf}
Define random event 
$$
\xi = \left\{\forall e\in[n]\; \forall t>0, |\bar w_t(e)-w(e)|\le \rad_t(e) \right\}.
$$
Then, there exist constants $c_{\rad}$ and $c_\delta$,
$$
\Pr[\xi] \ge 1-\delta.
$$
\end{lemma}
\begin{proof}
Hoeffding inequality and union bound.
\end{proof}



\begin{corollary}
\label{corr:conf}
$$
\xi \implies \forall t,\forall e\in[n] \; w_t^+(e) \ge w(e).
$$
$$
\xi \implies \forall t,\forall M\subseteq[n],\; w_t^+(M) \ge w(M).
$$
$$
\xi \implies \forall t,\forall b\in\diff[n]\; w_t^+(b) \ge w(b).
$$
\end{corollary}

\junk{
\begin{lemma}
Let $c\in \diff[n],d\in \diff[n]$. Let $\rad_t$ be a radius function.
Then,
\begin{equation}
\rad_t(c\cup d) = \rad_t(c)+\rad_t(d)-\rad_t(c\cap d).
\end{equation}
\end{lemma}

\begin{proof}
Let $c=(c_+,c_-)$ and $d=(d_+,d_-)$.
We have
\begin{align}
\rad_t(c\cup d) &= \rad_t(c_+\cup d_+) + \rad_t(c_-\cup d_-)\\
			  &= \rad_t(c_+)+\rad_t(d_+)-\rad_t(c_+\cap d_+)
			    +\rad_t(c_-)+\rad_t(d_-)-\rad_t(c_-\cap d_-)\\
			  &= \rad_t(c)+\rad_t(d)-\rad_t(c\cap d).
\end{align}
\end{proof}
}

\subsection{Properties of $\rad_t$}

\begin{lemma}
\label{lemma:rad-diff-simple}
Let $c\in \diff[n],d\in \diff[n]$.
Then
\begin{equation}
\rad_t(c\del d) = \rad_t(c)-\rad_t(c\cap d).
\end{equation}
\end{lemma}

\begin{proof}
Let $c=(c_+,c_-)$ and $d=(d_+,d_-)$.
We have
\begin{align*}
\rad_t(c\del d) &= \rad_t(c_+\del d_+)+\rad_t(c_-\del d_-)\\
			    &= \rad_t(c_+)-\rad_t(c_+\cap d_+)+\rad_t(c_-)-\rad_t(c_-\cap d_-)\\
			    &= \rad_t(c)-\rad_t(c\cap d).
\end{align*}

\end{proof}


\begin{lemma}
\label{lemma:diff-algebra-rad}

Let $C=(C_+,C_-)$ and $D=(D_+,D_-)$ be two diff-sets.
If $D \diffvalid C$, then
$$
\rad_t(C \oplus D) = \rad_t(C)+\rad_t(D)-2\rad_t(C_+ \cap D_-)-2\rad_t(C_- \cap D_+).
$$

In addition, if $\neg D \diffvalid C$, then
$$
\rad_t(C \ominus D) = \rad_t(C)+\rad_t(D)-2\rad_t(C_+ \cap D_+)-2\rad_t(C_- \cap D_-).
$$
\end{lemma}

\begin{proof}
We prove the first part of the lemma. The second part follows from the first part and the definition of $\neg D$.

By definition, we have $C\oplus D = ( (C_+ \cup D_+) \del (C_-\cup D_-), (C_-\cup D_-) \del (C_+\cup D_+) )$.
Hence, we have 
\begin{align}
\rad_t((C_+ \cup D_+) \del (C_-\cup D_-)) &= \rad_t(C_+ \cup D_+)-\rad_t( (C_+\cup D_+)\cap(C_-\cup D_-)) \\
							              &= \rad_t(C_+)+\rad_t(D_+)-\rad_t( (C_+\cup D_+)\cap(C_-\cup D_-)),						
\end{align}
where the second equality holds due to $C_+\cap D_+=\emptyset$ by the definition of $D\diffvalid C$.

Similarly, we have
$$
\rad_t((C_- \cup D_-) \del (C_+\cup D_+)) = \rad_t(C_-)+\rad_t(D_-)-\rad_t( (C_+\cup D_+) \cap(C_-\cup D_-)).
$$

Combine both equalities, we have
\begin{align}
  \rad_t(C \oplus D) &= \rad_t((C_+ \cup D_+) \del (C_-\cup D_))+\rad_t((C_- \cup D_-) \del (C_-\cap D_+))\\
  				     &=	\rad_t(C_+)+\rad_t(D_+)+\rad_t(C_-)+\rad_t(D_-)-2\rad_t( (C_+\cup D_+) \cap(C_-\cup D_-))\\
  				     &= \rad_t(C)+\rad_t(D)-2\rad_t( (C_+\cup D_+) \cap(C_-\cup D_-)).
\end{align}


\end{proof}


\begin{proof}
Suppose that $p_t=e$. 
Since $\B$ is an exchange class for $\M$, there exists an exchange set $c=(c_+,c_-)$ of $\B$
such that $e\in c_-$, $c_- \subseteq M_t$ and $c_+ \subseteq \tilde M_t$.

We decompose the proof into three cases.

\textbf{Case (1).} 
Suppose that $c \in\Bopt$. Then $w(c)>0$.
We have $\rad_t(e) \le \frac{1}{3}\Delta_e \le \frac{1}{3K}w(c)$.

In addition, by the choice of $p_t$,  we have $\forall g\in c_t, g\not=e$, $\rad_t(g) \le \rad_t(e)\le \frac{1}{3K}w(c)$.
Hence, $\rad_t(c) = \sum_{g\in c_t} \rad_t(g) \le \frac{|c_t|}{3K}w(c) \le \frac{1}{3}w(c)$.

Hence, $w_t(c) \ge w(c)-\rad_t(c) \ge \frac{2}{3}w(c) > 0$.
This means that $w_t(M_t \oplus c) = w_t(M_t)+w_t(c) > w_t(M_t)$.
Therefore, $M_t \not= \max_{M\in \M} \bar w_t(M)$.
This contradicts to the definition of $M_t$.

\textbf{Case (2).} 
Suppose that $c_t\not\in \Bopt$. Then, one of the following mutually exclusive cases must hold.

\textbf{Case (2.1).} 
$(e \in M_* \wedge e\in c_+)$ or $(e \not \in M_* \wedge e\in c_-)$. 
 
Let the decomposition of $M_* \ominus (M_t \oplus D \ominus c)$ on $\B$ be $b,b_1,\ldots,b_l$, which exists due to $\B$ is a diff-set class. 
Assume wlog that $e\in b$. 
We write $b=(b_+,b_-)$.
It is easy to see that $b \in \Bopt$. 

Define $\tilde D = (M_t \oplus D \ominus c)\ominus M_t$ and $D' = (M_t \oplus \tilde D \oplus b) \ominus M_t$.
By Lemma~\ref{lemma:diff-set-algebra}, we know that $\tilde D = D\ominus c$ and $D'=\tilde D\oplus b$.
We also write $\tilde D=(\tilde D_+, \tilde D_-)$ and $D'= (D'_+, D'_-)$.
By definition, we have 
\begin{align*}
	\tilde D_+ &= (D_+ \cup c_-)\del (D_- \cup c_+)\\
	           &= (D_+\cup c_-\del D_-) \cap (D_+\cup c_-\del c_+)\\
	           &= D_+ \cap (D_+\del c_-) \\
	           &= D_+\del c_+.
\end{align*}
By the same method, we are able to show that $\tilde D_-=D_-\del c_-$.
Therefore we have 
\begin{equation}
\label{eq:2.1.0.1}
\tilde D_+\subseteq D_+\quad\text{and}\quad\tilde D_-\subseteq D_-.
\end{equation}

First, we show that $\rad_t(c) \le \frac13 w(b)$.
Since $e\in E_t^3$, $e\in b$ and $b\in \Bopt$, we have $\rad_t(e) \le \frac{1}{3}\Delta_e \le \frac{1}{3K}w(b)$.
In addition, $\forall g\in c, g\not=e$, $\rad_t(g) \le \rad_t(e)\le \frac{1}{3K}w(b)$.
Hence, 
\begin{align}
\rad_t(c) &= \sum_{g\in c} \rad_t(g) \nonumber \\
		  & \le \frac{|c|}{3K}w(b) \nonumber \\
		  &\le \frac{1}{3}w(b).\label{eq:2.1.0.2}
\end{align}

%Consider the diff-set $D'=D\ominus c \oplus b$.
Now, we show that $\rad_t(\tilde D_+ \cap b_-)+\rad_t(\tilde D_- \cap b_+)+ \le \frac13 w(b)$.
Since Eq.~\eqref{eq:2.1.0.1}, we have $\forall g\in (\tilde D_+\cap b_-)\cup(\tilde D_-\cap b_+), g\not=e$, 
$\rad_t(g) \le \rad_t(e)\le \frac{1}{3K}w(b)$.
Note that $|\tilde D_+\cap b_-|+|\tilde D_-\cap b_+|\le |b_+|+|b_-| \le K$. 
Hence, 
\begin{align}
\rad_t(\tilde D_+ \cap b_-)+\rad_t(\tilde D_- \cap b_+) &= \sum_{g\in (\tilde D_+\cap b_-)\cup(\tilde D_-\cap b_+)} \rad_t(g) \nonumber\\
	& \le \frac{K}{3K}w(b) \nonumber \\
	& \le \frac{1}{3}w(b). \label{eq:2.1.0.3}
\end{align}

Then, we have
\begin{align}
\rad_t(D')-\rad_t(D) &= \rad_t(\tilde D \oplus b)-\rad_t(D) \\
					 &= \rad_t(\tilde D)+\rad_t(b)-2\rad_t(\tilde D_+\cap b_-)-2\rad_t(\tilde D_-\cap b_+)-\rad_t(D)
					 \label{eq:2.1.1}\\
					 &= \rad_t(D\ominus c)+\rad_t(b)-2\rad_t(\tilde D_+\cap b_-)-2\rad_t(\tilde D_-\cap b_+)-\rad_t(D)\\
					 &= \rad_t(D)+\rad_t(c)+\rad_t(b)-2\rad_t(D_+\cap c_+)-2\rad_t(D_-\cap c_-) \nonumber \\
					 &\quad -2\rad_t(\tilde D_+\cap b_-)-2\rad_t(\tilde D_-\cap b_+)-\rad_t(D)
					 \label{eq:2.1.2}\\
					 &= \rad_t(D)+\rad_t(c)+\rad_t(b)-2\rad_t(c_+)-2\rad_t(c_-) \nonumber \\
					 &\quad -2\rad_t(\tilde D_+\cap b_-)-2\rad_t(\tilde D_-\cap b_+)-\rad_t(D)
					 \label{eq:2.1.3}\\
					 &= \rad_t(b)-\rad_t(c)-2\rad_t(\tilde D_+\cap b_-)-2\rad_t(\tilde D_-\cap b_+),
\end{align}
where Eq.~\eqref{eq:2.1.1} and Eq.~\eqref{eq:2.1.2} follow from Lemma~\ref{lemma:diff-algebra-rad}, and 
Eq.~\eqref{eq:2.1.3} follows from Eq.~\eqref{eq:lemma-key-1}.

By the definition of $D$, we have that $w^+_t(D) \ge w^+_t(D')$. 
This means that 
\begin{align}
	\bar w_t(D)+\rad_t(D) &\ge \bar w_t(D')+\rad_t(D')\\
						  &= \bar w_t(D)-\bar w_t(c)+\bar w_t(b)+\rad_t(D').
\end{align}
By regrouping the above inequality, we have
\begin{align}
   \bar w_t(c) &\ge \bar w_t(b) + \rad_t(D')-\rad_t(D) \\
   			   &= \bar w_t(b)+\rad_t(b)-\rad_t(c)-2\rad_t(\tilde D_+\cap b_-)-2\rad_t(\tilde D_-\cap b_+)\\
   			   &\ge w(b)-\rad_t(c)-2\rad_t(\tilde D_+\cap b_-)-2\rad_t(\tilde D_-\cap b_+) \\
   			   &> w(b)-\frac13 w(b)-\frac23 w(b) \label{eq:2.1.4}\\
   			   &= 0,
\end{align}
where Eq.~\eqref{eq:2.1.4} follows from Eq.~\eqref{eq:2.1.0.2} and Eq.~\eqref{eq:2.1.0.3}.

This contradicts to the definition of $M_t$.

\textbf{Case (2.2).}
$(e \in M_* \wedge e\in c_-)$ or $(e \not \in M_* \wedge e\in c_+)$.

Let the decomposition of $M_* \ominus (M_t\oplus D)$ on $\B$ be $b,b_1,\ldots, b_l$.
Assume wlog that $e\in b$.
We write that $b=(b_+,b_-)$.
Note that $b\in \Bopt$ and hence $w(b)>0$.

Define $D'=(M_t\oplus D\oplus b)\ominus M_t$. By Lemma~\ref{lemma:diff-set-algebra}, we know that $D' = D\oplus b$.

First, we show that $|D\del D'| \le |b|$.
Let $C=D\del D'$ and write $C=(C_+,C_-)$.
We can bound $|C_+|$ as follows.
\begin{align*}
	C_+ &= D_+\del D'_+\\
		&= D_+\del \left((D_+\cup b_+) \del (D_-\cup b_-)\right)\\
		&= (D_+ \cap (D_-\cup b_-)) \cup (D_+ \del (D_+\cup b_+))\\
		&= D_+\cap b_-.
\end{align*}
Hence, we have $|C_+| \le |b_-|$.
Then, we move to bounding $|C_-|$
\begin{align*}
	C_- &= D_-\del D'_-\\
		&= D_-\del \left((D_-\cup b_-) \del (D_+\cup b_+)\right)\\
		&= (D_- \cap (D_+\cup b_+)) \cup (D_- \del (D_-\cup b_-))\\
		&= D_-\cap b_+.
\end{align*}
Thus $|C_-|\le |b_+|$ and we proved that $|D\del D'|\le |b|$.


Next, we show that $\rad_t(D \del D') \le \frac13 w(b)$.
Since $e\in E_t^3$, $e\in b$ and $b\in \Bopt$, we have $\rad_t(e) \le \frac{1}{3}\Delta_e \le \frac{1}{3K}w(b)$.
In addition, $\forall g\in (D \del D'), g\not=e$, $\rad_t(g) \le \rad_t(e)\le \frac{1}{3K}w(b)$.
Note that $|D\del D'| \le |b| \le K$.
Hence, $\rad_t(D\del D') = \sum_{g\in (D\del D')} \rad_t(g) \le \frac{K}{3K}w(b) \le \frac{1}{3}w(b)$.

We also note that
\begin{align}
w(D'\del D)-w(D\del D') &= w(D' \del D)+w(D'\cap D)-w(D\cap D')-w(D\del D')\\
						&= w(D')-w(D) \\
						&= w(b),
\end{align}
where we have repeatedly applied Lemma~\ref{lemma:weight-diff-simple}.

Then, we show that $w^+_t(D') > w^+_t(D)$.
\begin{align}
	w_t^+(D')-w_t^+(D) &= \bar w_t(D')-\bar w_t(D)+\rad_t(D')-\rad_t(D)\\
					   &= \bar w_t(D'\del D)-\bar w_t(D\del D')+\rad_t(D'\del D)-\rad_t(D\del D') \label{eq:2.2.x.1} \\
					   &\ge w(D'\del D)-w(D\del D')-2\rad_t(D\del D') \label{eq:2.2.x.2} \\
					   &= w(b)-2\rad_t(D\del D')\\
					   &> w(b)-\frac23 w(b) \\
					   &= \frac13 w(b) > 0,
\end{align}
where Eq.~\eqref{eq:2.2.x.1} follows from Lemma~\ref{lemma:rad-diff-simple} and Eq.~\eqref{eq:2.2.x.2} follows from
the fact that $\bar w_t(D'\del D)+\rad_t(D'\del D)\ge w(D'\del D)$ and that $\bar w_t(D\del D')+\rad_t(D\del D')\ge w(D\del D')$, under the random event $\xi$.

This contradicts to the fact that $D$ is chosen on round $t$.
\end{proof}

\begin{theorem}
Given a vector $\{w(1),\ldots, w(n)\}$, a budget $T>0$ and a collection of feasible sets $\M \subseteq 2^{[n]}$.
Let $\mathbb A$ be an arbitrary algorithm for $\M$-\Problem problem which uses at most $T$ samples.
There exists a vector $\{\tilde w(1), \ldots, \tilde w(n)\}$ such that $\mathbf H(\tilde w) \le 2\mathbf H(w)$ and satisfies the following property.
Consider the bandit problem with reward distributions  defined by $\Rew_e = \mathcal N(\tilde w(e), 1)$ for all $e\in[n]$, where $\mathcal N(\mu,\sigma^2)$ denotes Gaussian distribution with mean $\mu$ and variance $\sigma^2$.
The probability of error of $\mathbb A$ on this bandit problem satisfies
$$
\Pr\left[\out\not=M_* \right] \ge \exp\left(-\frac{T}{\mathbf H(w)}\right),
$$
where $\out$ is the output of $\mathbb A$ and $M_*=\argmax_{M\in \M} w(M)$ is the optimal set. 
In addition, vector $\{\tilde w(1),\ldots, \tilde w(n)\}$ differs from vector $\{w(1),\ldots,w(n)\}$ on exactly one index.
\end{theorem}

\begin{proof}
Fix $\M \subseteq 2^{[n]}$, $w(e)$ for all $e\in[n]$ and a fixed budget algorithm $\mathbb A$ for $\M$-\Problem problem.
Let $\sigma(1),\ldots,\sigma(n)$ be a permutation of $1,2,\ldots, n$ such that
 $\Delta_{\sigma(1)} \le \Delta_{\sigma(2)} \ldots \le \Delta_{\sigma(n)}$.
Define $L' = \argmax_{i\in [n]} i/\Delta_{\sigma(i)}^2$ and $L=\sigma(L')$.

Then, we construct hypothesis $H_0$ as follows
$$
H_0: \Rew_e = \mathcal N(w(e), 1) \quad \text{for all } e \in [n].
$$

We define random event $\mathcal C$ as follows.

We show that $\Pr_0[\mathcal C] \ge 1/2$.

We define random variables $X,Y,Z$ as follows
$$
X=\argmin_{i\in [L]\del \out} T_i, \quad Y=\argmin_{i\in [L]\cap \out} T_i \quad\text{and }
Z=\argmin_{i\in [L]} T_i, 
$$
where, for convenience, if $[L]\del \out=\emptyset$, we set $X=0$; and if $[L]\cap \out = \emptyset$, we set $Y=0$. 
By definition, we see that $X\not=Y$, $Z\in \{X,Y\}$ and $Z\not = 0$.
Now, by summing up all possible values of $X$, $Y$ and $Z$, we have
\begin{align*}
1/2 < \Pr_0[\mathcal C]
    = \sum_{\substack{x\in \{0,\ldots,L\}\\y\in \{0,\ldots, L\}\\x\not=y, z\in \{x,y\}}}
      \Pr_0[\mathcal C \cap \{X=x, Y=y, Z=z\}].
\end{align*}
Since a maximal is larger than an average, we see that there exists $x,y,z$ such that $x\not=y, z\in\{x,y\}$ and
\begin{equation}
\label{eq:lower-budget-z-prob}
\Pr_0[\mathcal C \cap \{X=x,Y=y,Z=z\}] \ge \frac{1}{4L(L+1)}.
\end{equation}
We point out that $x, y$ and $z$ are deterministic and only depends $\mathbb A$, $w$ and $\mathcal M$.
Now, depending on the value of $x,y$ and $z$, we divide our analysis into two cases.

\textbf{Case (1): $(z=x \wedge x\in M_*)$ or $(z=y \wedge y\not\in M_*)$.} 
Eq.~\eqref{eq:lower-budget-z-prob} implies that 
\begin{align*}
\Pr_0[\{X=x,Y=y,Z=z\}] &\ge \Pr_0[\mathcal C \cap \{X=x,Y=y,Z=z\}] \\
					   &\ge \frac{1}{4L(L+1)} \ge G. 
\end{align*}


First, let us assume that $z=x$ and $x\in M_*$.
By definition, we have $X\not\in\out$.
Notice that $x$ belong $M_*$. 
Therefore the event that $X=x$ and the assumption that $x\in M_*$ imply that $\out\not=M_*$.
This means that, if $z=x$ and $x\in M_*$, 
then $\Pr_0[\out\not=M_*] \ge \Pr_0[X=x] \ge G$.

Next, we assume that $z=y$ and $y\not \in M_*$.
Notice that $Y\in\out$ and $y\not\in M_*$.
Hence, the event $Y=y$ and the assumption that $y\not\in M_*$ imply that $\out\not=M_*$.
Therefore, if $z=y$ and $y\not\in M_*$, then
$\Pr_0[\out\not=M_*] \ge \Pr_0[Y=y] \ge  G$.

Therefore, we proved that, in Case (1), the probability of error of algorithm $\mathbb A$ is larger than $G$ under $H_0$.

\textbf{Case (2): $(z=x \wedge x\not \in M_*)$ or $(z=y \wedge y\in M_*)$.}
By definition, $Z$ is the arm with smallest number of samples among arms in $[L]$ and algorithm $\mathbb A$ uses at most $T$ samples. 
Therefore, we have
\begin{equation}
\label{eq:lower-budget-z-sample}
T_Z \le \frac{T}{L}.
\end{equation}

Then, we consider two cases separately.

\textbf{Case (2.1): $(z=x \wedge x\not \in M_*)$.}
We construct hypothesis $H_1$ as follows
$$
H_1: \Rew_x = \mathcal N(w(x)+\Delta_L+\varepsilon, 1) \quad\text{and}\quad
\Rew_e(e) = \mathcal N( w(e), 1) \quad \text{for all } e\not=y.
$$
Notice that, by the choice of $L$, we have $\Delta_x \ge \Delta_L$.
Hence we see that $w_1(M_x) = w_0(M_x)+\Delta_L \ge w_0(M_x)+\Delta_x = w_0(M_*)  = w_1(M_*)$.
Therefore, under $H_1$, $M_*$ is not the optimal set.
Now we bound the likelihood ratio $L_1(W)/L_0(W)$ as follows
\begin{align}
\frac{L_1(W)}{L_0(W)}
&= \prod_{i=1}^{T_Z} \exp\left(\frac{-(X_i-w(x)-\Delta_L)^2+(X_i-w(x))^2}{2}\right) \nonumber \\
&= \prod_{i=1}^{T_Z} \exp\big(\Delta_L(X_i-w(x))-\Delta_L^2\big) \nonumber \\
&= \exp\big(\Delta_L(K-T_Zw(x))-\Delta_L^2T_Z\big) \nonumber \\
&= \exp\big(\Delta_L(K-T_Zw(x))\big)\exp(-\Delta_L^2T_Z) \label{eq:lower-budget-case-1}.
\end{align}

Then we analyze the right-hand side of Eq.~\eqref{eq:lower-budget-case-1} as follows
\begin{align}
  \exp(-\Delta_Z^2T_Z) &\ge \exp(-\Delta_Z^2T/L) \label{eq:lower-bound-case-1-1.1} \\
					   &=\exp\left(-\frac{8}{16}\log(1/\theta)\right) \label{eq:lower-bound-case-1-1.2}\\
					   &= \theta^{1/2}\label{eq:lower-bound-case-1-1.3},
\end{align}
where Eq.~\eqref{eq:lower-bound-case-1-1.1} follows from the assumption that event $\mathcal S$ occurred, which implies that event $\mathcal A$ occurred and therefore $T_e \le 4t_e^*$; Eq.~\eqref{eq:lower-bound-case-1-1.2} follows from the definition of $t_e^*$.

Then, we bound the first term on the right-hand side of Eq.~\eqref{eq:lower-bound-case-1} as follows
\begin{align}
	\exp\big(\Delta_e(2K-2p_eT_e)\big) & \ge \exp\left(-2\Delta_e\sqrt{t_e^*\log(1/\theta)}\right) \label{eq:lower-bound-case-1-2.1}\\
								       & = \exp\left(-\frac{2}{\sqrt{4}}\log(1/\theta)\right) \label{eq:lower-bound-case-1-2.2}\\
								       &=\theta^{1/2},  \label{eq:lower-bound-case-1-2.3}
\end{align}
where Eq.~\eqref{eq:lower-bound-case-1-2.1} follows from the assumption that event $\mathcal S$ occurred, which implies that event $\mathcal C$ and therefore $|2K-2p_eT_e| \le \sqrt{t_e^*\log(1/\theta)}$; 
Eq.~\eqref{eq:lower-bound-case-1-2.2} follows from the definition of $t_e^*$.

\textbf{Case (2.2): $y\not\in M_*$.}
By definition of $y$, we see that the event $Y=y$ implies that $y\not\in\out$ and therefore $\out\not=M_*$.
On the other hand, using Eq.~\eqref{eq:lower-budget-z-prob}, we have
$$\Pr_0[Y=y] \ge \Pr_0[\mathcal C \cap \{Y=y, Z=0\} ] = \frac{1}{2(L+1)^2}.$$
This gives that $\Pr_0[\out\not=M_*] \ge \frac{1}{2(L+1)^2} \ge A$.


\end{proof}


\begin{lemma}
\label{lemma:pac-correct}
If Algorithm~\ref{algo:pac} stops, then $w(M_*)-w(\out) \le \epsilon$.
\end{lemma}

\begin{proof}
Suppose that $\out \not= M_*$.
Suppose that the algorithm stops on round $T$, we know that $\out = M_T$.
Consider the diff-set $D=M_* \ominus M_T$ and the diff-set $D_T$ as defined in Step~\ref{algo:step:D} of Algorithm~\ref{algo:pac}.
By Lemma~Z, we see that 
\begin{equation}
w_T^+(D_T) = \max_{C: C \diffvalid M_T} w_T^+(C) \ge w_T^+(D).
\label{eq:pac-correct-0}
\end{equation}
On the other hand, the stopping condition of Algorithm~\ref{algo:pac} gives that
\begin{align}
\epsilon &\ge \tilde w_T(\tilde M_T)-\tilde w_T(M_T) \nonumber \\
		 &= w_T^+(D_T) \ge w_T^+(D) \label{eq:pac-correct-1} \\
		 &\ge w(D) = w(M_*)-w(M_T), \label{eq:pac-correct-2}
\end{align}
where Eq.~\eqref{eq:pac-correct-1} follows from Eq.~\eqref{eq:pac-correct-0}; Eq.~\eqref{eq:pac-correct-2} follows from the assumption that event $\xi$ occurred.


\end{proof}

\bibliography{bandit}
\bibliographystyle{plain}
\end{spacing}
\end{document}