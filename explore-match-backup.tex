\documentclass{article}
\title{Pure Exploration of Combinatorial Bandits}
\author{Shouyuan Chen}
\date{\today}
%%%%%%%%%%%%%%%%%%%%%%%%%%%%%%%%%%%%%%%%%%%%%%%%%%%%%%%%%%%%%

% Change "article" to "report" to get rid of page number on title page
\usepackage{amsmath,amsfonts,amsthm,amssymb}
\usepackage{setspace}
\usepackage{Tabbing}
\usepackage{fancyhdr}
\usepackage{lastpage} 
\usepackage{extramarks}
\usepackage{chngpage}
\usepackage{soul,color}
\usepackage{graphicx,float,wrapfig}
\usepackage{afterpage}
\usepackage{abstract}
\usepackage{hyperref}

% In case you need to adjust margins:
\topmargin=-0.45in
\evensidemargin=0in
\oddsidemargin=0in
\textwidth=6.5in
\textheight=9.0in
\headsep=0.25in

% Setup the header and footer
\pagestyle{fancy}
%\lhead{\LatexerName}
%\chead{\LectureClassName: \LectureTitle}
%\rhead{\LectureDate}
%\lfoot{\lastxmark}
%\cfoot{}
\rfoot{Page\ \thepage\ of\ \pageref{LastPage}}
\renewcommand\headrulewidth{0.4pt}
\renewcommand\footrulewidth{0.4pt}


%%%%%%%%%%%%%%%%%%%%%%%%%%%%%%%%%%%%%%%%%%%%%%%%%%%%%%%%%%%%%
% Some tools

\newtheorem{define}{Definition}
\newtheorem{example}{Example}
\newtheorem{lemma}{Lemma}
\newtheorem{corollary}{Corollary}

\newcommand{\M}{\mathcal M}
\newcommand{\mmatch}{\mathcal M_{\mathsf{MATCH}}}
\newcommand{\mtop}{\mathcal M_{\mathsf{TOP}m}}
\newcommand{\mbandit}{\mathcal M_{\mathsf{BANDIT}m}}

\newcommand{\diff}{\mathsf{diff}}
\newcommand{\diffvalid}{\prec}
\newcommand{\B}{\mathcal B}
\newcommand{\C}{\mathcal C}
\newcommand{\del}{\backslash}

\newcommand{\RR}{\mathbb R}


\newcommand{\Bopt}{\mathcal B_{\mathsf{opt}}}
\newcommand{\Bmatch}{\mathcal B_{\mathsf{MATCH}}}
\newcommand{\Btop}{\mathcal B_{\mathsf{TOP}m}}
\newcommand{\Bbandit}{\mathcal B_{\mathsf{BANDIT}m}}

\DeclareMathOperator{\rank}{rank}
\DeclareMathOperator{\rad}{rad}
%%%%%%%%%%%%%%%%%%%%%%%%%%%%%%%%%%%%%%%%%%%%%%%%%%%%%%%%%%%%%

\begin{document}
\begin{spacing}{1.1}
\newpage

\maketitle

\section{Preliminaries}

\subsection{Problems}

Let $n$ be the number of base arms. 
Let $\M \subseteq 2^{[n]}$ be the set of super arms. 


In this note, we consider the following cases of $\M$.

\begin{example}[Explore-$m$]
$\mtop(n)=\{M \subseteq [n] \;|\; |M|=m\}$.
This corresponds to finding the top $m$ arms from $[n]$.
\end{example}

\begin{example}[Explore-$m$-bandits]
Suppose $n=mk$. Then $\mbandit(n)$ contains all subsets $M \subseteq [n]$ with size $m$, such that 
$$ 
M\cap \{ik+1,\ldots, (i+1)k\} = 1, \quad \text{for all } i \in \{0,\ldots, m-1\}.
$$ 
This corresponds to finding the top arms from $m$ bandits, where each bandit has $k$ arms.
\end{example}

\begin{example}[Perfect Matching]
Let $G=(V,E)$ be a bipartite graph and $|E|=n$. 
For simplicity, let each edge $e\in E$ corresponds to a unique integer $i\in [n]$, and vice versa. 
Then $\mmatch(n,G)$ contains all subsets $M \subseteq [n]$ such that $M$ corresponds to a perfect matching in $G$.
\end{example}


\subsection{Diff-Sets}

\begin{define}[Diff-set]
An $n$-diff-set (or diff-set in short) is a pair of sets $c=(c_+,c_-)$, where $c_+\subseteq[n]$ and $c_-\subseteq [n]$.
\end{define}


\begin{define}[Difference of sets]
Given any $M_1\subseteq[n],M_2\subseteq[n]$. We define $M_1\ominus M_2 \triangleq C$, where $C=(C_+,C_-)$ is a diff-set and
$C_+ = M_1 \del M_2$ and $C_- = M_2\del M_1$.
\end{define}

\begin{define}
Denote $\diff[n]$ be the set of all possible $n$-diff-sets.
\end{define}

\begin{define}[Set operations of diff-sets] 
Let $C=(C_+,C_-), D=(D_+,D_-)$ be two diff-sets. 
We define $C\cup D \triangleq (C_+\cup D_+, C_-\cup D_-)$,
$C\cap D \triangleq (C_+\cap D_+, C_-\cap D_-)$
and $C\del D \triangleq (C_+\del D_+, C_-\del D_-)$.
\end{define}

\begin{define}[Valid diff-set]
Given a set $M \subseteq [n]$ and a diff-set $C=(C_+,C_-)$, we call $C$ a \emph{valid diff-set} for $M$, iff $C_+ \cap M = \emptyset$ and $C_- \subseteq M$.
In this case, we denote $C\diffvalid M$.
\end{define}

\begin{define}[Application of diff-sets]
Given any $M \subseteq [n]$ and $C \in \diff[n]$ such that $C \diffvalid M$.
Then we define operators $\oplus$ and $\ominus$ such that $M \oplus C \triangleq M\cup C_+ \del C_-$ and $M \ominus C \triangleq M\cup C_- \del C_+$.
\end{define}

\begin{define}[diff-set class]
\label{define:diff-class}
Given $\M \subseteq 2^{[n]}$. $\B \subseteq \diff[n]$ is a diff-set class for $\M$, if the following hold.
\begin{enumerate} 
\item $(\emptyset,\emptyset)\not\in \B$.
\item For all $M\in \M$ and for all $b\in \B$, if $b\diffvalid M$, then $M\oplus b \in \M$.
\item For all $M_1 \in \M$ and $M_2\in \M$, where $M_1\not=M_2$. 
	  There exists a sequence of diff-sets $b_1 \in \B,\ldots,b_k \in \B$, such that 
	  $M_2 = M_1\oplus b_1 \ldots \oplus b_k$.
\end{enumerate}
\end{define}

\begin{define}[Rank of diff-set class]
Let $\B \subseteq [n]$ be a diff-set class for some $\M$. We define
$$
\rank(\B) \triangleq \max_{(b_+,b_-) \in \B} |b_+|+|b_-|.
$$
\end{define}

\begin{example}[diff-set class for Explore-$m$]
One diff-set class $\B$ for $\mtop(n)$ is given by
$$
\B=\{(\{b_1\},\{b_2\}) \;|\; b_1\not=b_2, b_1\in[n], b_2\in[n]\}.
$$
Proof omitted.
Further, we see that $\rank(\B)=2$.
\end{example}

\begin{example}[diff-set class for Explore-$m$-badit]
Let $n=mk$.
One diff-set class $\B$ for $\mbandit(n)$ is given by
$$
\B=\{(\{b_1\},\{b_2\}) \;|\; b_1\not=b_2, \exists i\in\{0,\ldots,k-1\}, b_1\in \{ik+1,\ldots, (i+1)k\}, b_2\in\{ik+1,\ldots, (i+1)k\}\}.
$$
Proof omitted.
Further, we see that $\rank(\B)=2$.
\end{example}


\begin{example}[diff-set class for Perfect Matching]
One diff-set class $\B$ for $\mmatch(n, G)$ is the set of all augmenting cycles of $G$. 
More specifically,
$$
\B=\{(b_+,b_-) | b_+\cup b_- \text{ is a cycle of } G\}.
$$

Note $\rank(\B)\le n$.
\end{example}

\subsection{Weights and confidence bounds}

\begin{define}[Weight functions]
Define function $w: [n] \rightarrow \RR^+$ which represents the weight of each base arm. 
Further, we slight abuse the notations, and extend the definition of $w$ to diff-sets and sets as follows.
\begin{enumerate}
\item For all $M \subseteq [n]$, we denote $w(M) = \sum_{e\in M} w(e)$.
\item For all $b=(b_+,b_-) \in \diff[n]$, we denote $w(b) = \sum_{e\in b_+} w(e)-\sum_{e\in b_-}w(e)$.
\end{enumerate}
\end{define}


\begin{lemma}
Let $c\in \diff[n],d\in \diff[n]$. Let $w$ be a weight function.
Then,
\begin{equation}
w(c\cup d) = w(c)+w(d)-w(c\cap d).
\end{equation}
\end{lemma}

\begin{proof}
Let $c=(c_+,c_-)$ and $d=(d_+,d_-)$.
We have
\begin{align}
w(c\cup d) &= w(c_+\cup d_+)-w(c_-\cup d_-)\\
           &= w(c_+)+w(d_+)-w(c_+\cap d_+)-w(c_-)-w(d_-)+w(c_- \cap d_-)\\
           &= w(c)+w(d)-(w(c_+\cap d_+)-w(c_-\cap d_-))\\
           &= w(c)+w(d)-w(c\cap d).
\end{align}
\end{proof}

\begin{define}[Mean weight $\bar w_t$, sample size $n_t$]
Given $t>0$. 
Define $\bar w_t$ be a weight function such that, for all $e\in[n]$, $\bar w_t(e)$ equals to the empirical mean of $e$ up to round $t$.
Let $n_t: [n] \rightarrow \mathbb N$, such that $n_t(e)$ equals to number of plays of base arm $e$ up to round $t$.
\end{define}

\begin{define}[Confidence radius $\rad_t$]
Given $n$ and $t>0$.
Define $\rad_t:[n]\rightarrow \RR^+$ satisfying, for all $e\in[n]$,
$$
\rad_t(e) = c_{\rad}\log\left(\frac{c_\delta nt^2}\delta\right)\frac{1}{\sqrt{n_t(e)}},
$$
where $c_{\rad} > 0$  and $c_\delta>0$ are some universal constants (specify later) and $\delta > 0$ is a parameter.

We extend the notation of $\rad_t$ to diff-sets and sets as follows.
\begin{enumerate}
\item For all $M \subseteq [n]$, $\rad_t(M) \triangleq \sum_{e\in M} \rad_t(e)$.
\item For all $b=(b_+,b_-)\in \diff[n]$, $\rad_t(b) \triangleq \rad_t(b_+)+\rad_t(b_-)$.
\end{enumerate}

\end{define}

\begin{define}[UCB $w_t^+$]
Define $w^+_t: [n] \rightarrow \RR^+$, s.t., for all $e\in[n]$,  
$$ w^+_t(e)=\bar w_t(e)+\rad_t(e).$$

We extend the notation of $w_t^+$ to diff-sets and sets as follows.
\begin{enumerate}
\item For all $M \subseteq [n]$, $w_t^+(M) \triangleq \bar w_t(M)+\rad_t(M)$.
\item For all $b=(b_+,b_-)\in \diff[n]$, $w_t^+(b) \triangleq \bar w_t(b)+\rad_t(b)$.
\end{enumerate}

\end{define}

\begin{lemma}
\label{lemma:conf}
Define random event 
$$
\xi = \left\{\forall e\in[n]\; \forall t>0, |\bar w_t(e)-w(e)|\le \rad_t(e) \right\}.
$$
Then, there exist constants $c_{\rad}$ and $c_\delta$,
$$
\Pr[\xi] \ge 1-\delta.
$$
\end{lemma}
\begin{proof}
Hoeffding inequality and union bound.
\end{proof}

\begin{corollary}
\label{corr:conf}
$$
\xi \implies \forall t,\forall e\in[n] \; w_t^+(e) \ge w(e).
$$
$$
\xi \implies \forall t,\forall M\subseteq[n],\; w_t^+(M) \ge w(M).
$$
$$
\xi \implies \forall t,\forall b\in\diff[n]\; w_t^+(b) \ge w(b).
$$
\end{corollary}



\section{Admissible diff-set classes}

\begin{define}

\begin{enumerate}
\item For all $e\in[n]$ and $c \not\in \Bopt$. If $(e \in M_* \wedge e\in c_-)$ or $(e \not \in M_* \wedge e\in c_+)$,
then there exists $b\in \Bopt$ such that $e\in b$ and $(c \oplus b) \in \B$.
\end{enumerate}

\end{define}

\begin{lemma}
Given a bipartite graph $G=(V,E)$, $\Bmatch(G)$ is an admissible diff-set class.
\end{lemma}

\begin{proof}
If $(e \in M_* \wedge e\in c_-)$ or $(e \not \in M_* \wedge e\in c_+)$.
Let $c \in \Bmatch(G) \del \Bopt$ be a diff-set.
Since $c$ represents an augmenting cycle of $G$,
we can write 
$c=\{c_0,c_1,\ldots,c_{l-1}\}$, where, for all $i\in\{0,\ldots,l-1\}$,
$c_i$ represents an edge of $G$, $(c_{i},c_{i+1})$ has a common vertex (with $c_l=c_0$), and $c_i \in c_+$ if $i \mod 2=0$.

Let $M$ be an arbitrary set such that $c\diffvalid M$. 
Let $M'=M\oplus c$.

\textbf{Case (1).} Suppose that $e\in M_*$ and $e\in c_-$. 
Since $e\not\in M'$, therefore $M'\not= M_*$.
Consider the subgraph induced by $M'\cup M_*$.
By the property of matching, $M'\cup M_*$ is a union of several disjoint cycles with even lengths.
Let $t$ be such a cycle and $e\in t$.
We write $t=\{t_0,\ldots, t_{m-1}\}$ and assume without generality that $t_0=e$.
Then, we define $b_+=\{t_0,t_2,\ldots,t_{m-2}\}$ and $b_-=\{t_1,t_3,\ldots,t_{m-1}\}$.
Clearly, $b_+\subseteq M_*$ and $b_-\cap M_* = \emptyset$. 
Let $b=(b_+,b_-)$.
We see that $b\in \Bopt$.

We remain to show that $c\oplus b \in \Bmatch(G)$, i.e. $c\oplus b$ represents an augmenting cycle. 


%By Property X, there exists $b\in \Bopt$ such that $b\diffvalid M'$ and $b$. 


\end{proof}



\section{Proof}

Unless specified, we shall assume the random event $\xi$ (defined in Lemma~\ref{lemma:conf}) holds in all the following proofs.


\begin{lemma}
\label{lemma:correct}
For any $t>0$, if the algorithm terminates on round $t$, then $M_t=M_*$.
\end{lemma}

\begin{proof}
Suppose $M_t \not= M_*$. Then $w(M_*)>w(M_t)$. 
Then, there exists $b \in \B$ such that $b \diffvalid M_t$ and $w(b)>0$.
On the other hand, by Corollary~\ref{corr:conf}, we have $w_t^+(b) > w(b)$.
Hence $w_t^+(b)>0$. This contradicts to the stopping condition of our algorithm.
\end{proof}

\begin{lemma}
For any $t>0$.
If $e\in E_t^2$, then $p_t\not= e$.
\end{lemma}

\begin{proof}
Suppose that $p_t=e$. Consider the chosen diff-set $c_t=(c_+,c_-)$ of round $t$.
Denote $K=\rank(\B)$. Note that $|c_t| \le K$.

\textbf{Case (1).} 
Suppose that $c_t\in\Bopt$. Then $w_t(c_t)>0$.
Since $e\in E_t^2$, we have $\rad_t(e) \le \frac{1}{2}\Delta_e \le \frac{1}{2K}w(c_t)$.
In addition, $\forall g\in c_t, g\not=e$, $\rad_t(g) \le \rad_t(e)\le \frac{1}{2K}w(c_t)$.
Hence, $\rad_t(c_t) = \sum_{g\in c_t} \rad_t(g) \le \frac{|c_t|}{2K}w(c_t) \le \frac{1}{2}w(c_t)$.

Hence, $\bar w_t(c_t) \ge w_t(c_t)-\rad_t(c_t) \ge \frac{1}{2}w(c_t) > 0$.
This means that $\bar w_t(M_t \oplus c_t) = \bar w_t(M_t)+\bar w_t(c_t) > \bar w_t(M_t)$.
Therefore, $M_t \not= \max_{M\in \M} \bar w_t(M)$.
This contradicts to the definition of $M_t$.

\textbf{Case (2).} 
Suppose that $c_t\not\in \Bopt$. Then, one of the following mutually exclusive cases must hold.

\textbf{Case (2.1).} 
$(e \in M_* \wedge e\in c_+)$ or $(e \not \in M_* \wedge e\in c_-)$. 

By Lemma~X, there exists $d\in \Bopt$ such that $d=(d_+,d_-)$ and $e\in (c_+\cap d_+)\cup(c_-\cap d_-)$.

First, we show that $\rad_t(c_t) \le \frac12 w_t(d)$.
Since $e\in E_t^2$, $e\in d_+\cup d_-$ and $d\in \Bopt$, we have $\rad_t(e) \le \frac{1}{2}\Delta_e \le \frac{1}{2K}w(d)$.
In addition, $\forall g\in c_t, g\not=e$, $\rad_t(g) \le \rad_t(e)\le \frac{1}{2K}w(d)$.
Hence, $\rad_t(c_t) = \sum_{g\in c_t} \rad_t(g) \le \frac{|c_t|}{2K}w(d) \le \frac{1}{2}w(d)$.

Then, we show that $\bar w_t(c_t) > 0$.
Since, $c_t$ is chosen on round $t$. 
By definition, we know that $w^+_t(c_t) \ge w^+_t(d)$.
This means that $\bar w_t(c_t)+\rad_t(c_t) \ge \bar w_t(d)+\rad_t(d)$.
Therefore,
\begin{align}
	\bar w_t(c_t) & \ge \bar w_t(d_t)+\rad_t(d)-\rad_t(c_t)\\
				  & \ge w(d)-\rad_t(c_t)\\
				  & \ge \frac{1}{2} w(d) \\
				  & > 0.
\end{align}

We have shown that $\bar w_t(c_t)>0$. And by definition, $c_t \diffvalid M_t$. Hence, $\bar w_t(M_t\oplus c_t) = \bar w_t(M_t)+\bar w_t(c_t) > \bar w_t(M_t)$. Contradiction.


\textbf{Case (2.2).}
$(e \in M_* \wedge e\in c_-)$ or $(e \not \in M_* \wedge e\in c_+)$.

By Lemma~Y, there exists $b\in \Bopt$ such that $e\in b$ and $(c_t\oplus b) \in \B$. Define $d=c_t\oplus b$. Let $d=(d_+,d_-)$.

First, we show that $\rad_t(c_t) \le \frac{1}{2}w(b)$. Since $e\in E_t^2$, $e\in b$ and $b\in \Bopt$, we have $\rad_t(e) \le \frac{1}{2}\Delta_e \le \frac{1}{2K}w(b)$.
In addition, $\forall g\in c_t, g\not=e$, $\rad_t(g) \le \rad_t(e)\le \frac{1}{2K}w(b)$.
Hence, $\rad_t(c_t) = \sum_{g\in c_t} \rad_t(g) \le \frac{|c_t|}{2K}w(b) \le \frac{1}{2}w(b)$.

Then, we show that $w^+_t(d) > w^+_t(c_t)$.
Note that
\begin{align}
w(d\del c_t)-w(c_t\del d) &= w(d\del c_t)+w(d\cap c_t)-w(d\cap c_t)-w(c_t\del d)\\
						  &= w(d)-w(c_t) \\
						  &= w(b).
\end{align}
We have
\begin{align}
w^+_t(d)-w^+_t(c_t) &= \bar w_t(d)-\bar w_t(c_t)+\rad_t(d)-\rad_t(c_t)\\
					&= \bar w_t(d\del c_t)-\bar w_t(c_t\del d)+\rad_t(d)-\rad_t(c_t)\\
					&= \bar w_t(d\del c_t)-\bar w_t(c_t\del d)+\rad_t(d \del c_t)-\rad_t(c_t \del d)\\
					&\ge w(d\del c_t)-\bar w_t(c_t \del d)-\rad_t(c_t \del d)\\
					&\ge w(d\del c_t)-w(c_t \del d)-2\rad_t(c_t \del d)\\
					&\ge w(b)-2\rad_t(c_t) \\
					&> w(b)-w(b)\\
					&= 0.
\end{align}

This contradicts to the fact that $c_t$ is chosen on round $t$.
\end{proof}


%\bibliography{bandit}
%\bibliographystyle{abbrv}
\end{spacing}
\end{document}